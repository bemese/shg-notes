\section{Coding}\label{code}
In this Appendix we reproduce all the quantities  that should be coded.

Eqs.~\eqref{calvimchiewn}, \eqref{calvimchie2wn}, and \eqref{calvimchiwn}
\eqref{calvimchi2wn}
\begin{align}\label{c-calvimchiewn}
\mathrm{Im}[\chi_{e,\rma\rmb\rmc,\go}^{S,\ell}] =
-\frac{\pi |e|^3}{2\hbar^2}\sum_{vc\bfk}\sum_{l\neq(v,c)}\frac{1}{\omega^{\gs}_{cv}}
\left[
\frac{\mathrm{Im}[\mathcal{V}^{\gs,\text{a},\ell}_{lc}\{r^{\rmb}_{cv}r^{\rmc}_{vl}\}]}
{(2\go^\gs_{cv}-\go^\gs_{cl})} 
-\frac{\mathrm{Im}[\mathcal{V}^{\gs,\text{a},\ell}_{vl}\{r^{\rmc}_{lc}r^{\rmb}_{cv}\}]}
{(2\go^\gs_{cv}-\go^\gs_{lv})}
\right]\gd(\go^\gs_{cv}-\go),
\end{align}  
\begin{align}\label{c-calvimchie2wn}
\mathrm{Im}[\chi_{e,\rma\rmb\rmc,2\go}^{S,\ell}] =
-\frac{\pi |e|^3}{2\hbar^2}\sum_{vc\bfk}\frac{4}{\omega^{\gs}_{cv}}
\left[
\sum_{v'\ne
  v}\frac{\mathrm{Im}[\mathcal{V}^{\gs,\text{a},\ell}_{vc}\{r^{\rmb}_{cv'}r^{\rmc}_{v'v}\}]}
{2\go^\gs_{cv'}-\go^\gs_{cv}}
- \sum_{c'\ne
  c}\frac{\mathrm{Im}[\mathcal{V}^{\gs,\text{a},\ell}_{vc}\{r^{\rmc}_{cc'}r^{\rmb}_{c'v}\}]}
{2\go^\gs_{c'v}-\go^\gs_{cv}}
\right]\gd(\go^\gs_{cv}-2\go),
\end{align}
\begin{align}\label{c-calvimchiwn}
\mathrm{Im}[\chi_{i,\text{a}\text{b}\text{c},\omega}^{S,\ell}]
= -\frac{\pi\vert e\vert^3}{2\hbar^2}\sum_{cv\mathbf{k}}\frac{1}{(\omega^{\gs}_{cv})^{2}}
\left(
\mathrm{Re}\left[r^{\text{b}}_{cv}\left(\mathcal{V}^{\gs,\text{a},\ell}_{vc}\right)_{;k^{\text{c}}}\right]
+\frac{\,\mathrm{Re}\left[\mathcal{V}^{\gs,\text{a},\ell}_{vc}r^{\text{b}}_{cv}\right]
\Delta^{\text{c}}_{cv}}{\omega^{\gs}_{cv}} 
\right)\delta(\omega^{\gs}_{cv}-\omega),
\end{align}
and
\begin{align}\label{c-calvimchi2wn}
\mathrm{Im}[\chi_{i,\text{a}\text{b}\text{c},2\omega}^{S,\ell}] 
=
 -\frac{\pi \vert
   e\vert^{3}}{2\hbar^2}\sum_{vc\mathbf{k}}\frac{4}{(\omega^{\gs}_{cv})^{2}}
\left(\mathrm{Re}\left[\mathcal{V}^{\gs,\text{a},\ell}_{vc}\left(r^{\text{b}}_{cv}\right)_{;k^{\text{c}}}
\right] -
\frac{2\,\mathrm{Re}\left[\mathcal{V}^{\gs,\text{a},\ell}_{vc}r^{\text{b}}_{cv}\right]
\Delta^{\text{c}}_{cv}}{\omega^{\gs}_{cv}}\right)\delta(\omega^{\gs}_{cv}-2\omega).
\end{align}
To evaluate above expressions we need the following ($m_e=1$):
\begin{align}\label{}
\bfv^\lda_{nm}(\bfk) 
=(1/m_e)\bfp_{nm}(\bfk)+\bfv^\nl_{nm}(\bfk)
=\bfp_{nm}(\bfk)+\bfv^\nl_{nm}(\bfk)
,
\end{align}
that
 includes the local and nonlocal parts of the pseudopotential. They
 correspond to the following files:\\
$\bullet$ $\bfp_{nm}(\bfk)\to$ \verb=me_pmn_*=\\
$\bullet$ $\bfv^\nl_{nm}(\bfk)\to$ \verb=me_vnlnm_*=\\
where the \verb=nm= or \verb=mn= order in the files is irrelevant, and
ought to be fixed just for the {\it biuty} of it.
Option \verb=-n= in \verb=all_responses.sh= does
\begin{enumerate}
\item 
 \verb=> cp me_pmn_* me_pmn_*.o= 
\item adds \verb=me_pmn_*= and \verb=me_vnlnm_*= into
  \verb=me_pmn_*= 
\item calculates the response
\item \verb=> mv me_pmn_*.o me_pmn_*=
\end{enumerate}
so   
$\bfv^\lda_{nm}(\bfk)$
is available for the calculation of the response, and with it we calculate
(Eqs.~\eqref{chon.9} and \eqref{chon.10}),
\begin{align}\label{c-chon.98}
\bfv^\gs_{nm}(\bfk)
&=
\left(1+\frac{\gS}{\go_c(\bfk)-\go_v(\bfk)}\right)\bfv^\lda_{nm}(\bfk)
\quad\quad n\notin D_m
\nonumber\\
\bfv^\gs_{nn}(\bfk)
&=
\bfv^\lda_{nn}(\bfk)
\nonumber\\
\bfr_{nm}(\bfk)&=\frac{\bfv^\gs_{nm}(\bfk)}{i\go^\gs_{nm}(\bfk)}
=\frac{\bfv^\lda_{nm}(\bfk)}{i\go^\lda_{nm}(\bfk)}
\quad\quad n\notin D_m
.
\end{align}   
If option \verb=-n= is not chosen, then the contribution of $\bfv^\nl_{nm}(\bfk)$
is neglected in the calculation of any response. Obviously, in this
case the code only uses \verb=me_pmn_*= without adding \verb=me_vnlnm_*= 

We need Eq.~\eqref{a.1} and \eqref{a.2}
\begin{align}\label{c-a.1}
\calv^{\gs,\rma,\ell}_{nm}
&=
\calv^{\lda,\rma,\ell}_{nm}
+
\calv^{\cals,\rma,\ell}_{nm}
\nonumber\\
\left(
\calv^{\gs,\rma,\ell}_{nm}
\right)_{;k^\rmb}
&=
\left(
\calv^{\lda,\rma,\ell}_{nm}
\right)_{;k^\rmb}
+
\left(
\calv^{\cals,\rma,\ell}_{nm}
\right)_{;k^\rmb}
.
\end{align}
 The first LDA term is
\begin{align}\label{c-a.2}
\calv^{\lda,\rma,\ell}_{nm}
&=
\frac{1}{2}\sum_q\left(
v^{\lda,\rma}_{nq}\calc^\ell_{qm}+\calc^\ell_{nq} v^{\lda,\rma}_{qm}
\right)
.
\end{align} 
If option \verb=-n= is not chosen in \verb=all_responses.sh=
Eq.~\eqref{c-a.2}
 is
not calculated and\\
$\bullet$ $\calv^{\lda,\rma,\ell}_{nm}\to$ \verb=me_cpmn_*=\\  
If option \verb=-n= is chosen Eq.~\eqref{c-a.2}
 must be calculated in
\verb=set_input_ascii.f90=. We mention that
$\calv^{\lda,\rma,\ell}_{nm}$ can be computed directly,\cite{nicolaspc}
avoiding the sum over the full set of bands $q$, however we chose to
compute Eq.~\eqref{c-a.2}, which is done in
\verb=functions.f90=.
Then, we need 
Eq.~\eqref{eni.4}
\begin{align}\label{c-eni.4}
\calc^\ell_{nm}(\bfk)&=
\sum_{\bfG,\bfG'} A^*_{n\bfk}(\bfG')  A_{m\bfk}(\bfG)
\gd_{\bfG_\parallel \bfG'_\parallel}
f_\ell(G_\perp-G'_\perp)
\nonumber\\
\calc^\ell_{mn}(\bfk)&=
\big(\calc^\ell_{nm}(\bfk)\big)^*
,
\end{align} 
which is coded in \verb=sub_pmn_ascii.f90= within the same subroutine of $\calbv^\ell_{nm}$
calculated with Eq.~\eqref{eni.2}. However, Sean out of the blue, call
it \verb=me_cfmn_*= in \verb=run_tiniba.sh=,
 and Darwin won (what else? ID??), 
thus I call it \verb=cfMatElem= in \verb=SRC_1setinput=. ID would call
it  \verb=ccMatElem=
but long live CD!

 The second LDA term is
\begin{align}\label{c-a.2n}
\left(\calv^{\lda,\rma,\ell}_{nm}\right)_{;k^\rmb}
&=
\frac{1}{2}\sum_q\left(
(v^{\lda,\rma}_{nq})_{;k^\rmb}\calc^\ell_{qm}
+ 
v^{\lda,\rma}_{nq}(\calc^\ell_{qm})_{;k^\rmb}
+
(\calc^\ell_{nq})_{;k^\rmb} v^{\lda,\rma}_{qm}
+
\calc^\ell_{nq} (v^{\lda,\rma}_{qm})_{;k^\rmb}
\right)
,
\end{align} 
where\\
$\bullet$ for $n\ne m$\\
Eq.~\eqref{a.3}
\begin{align}\label{c-a.3}
(v^{\lda,\rma}_{nm})_{;k^\rmb}
&=  
im_e\left(\gD^b_{nm}r^\rma_{nm}
+ 
\go^\lda_{nm}(r^\rma_{nm})_{;k^\rmb}
\right)
\nonumber\\
(v^{\lda,\rma}_{mn})_{;k^\rmb}
&=
\left((v^{\lda,\rma}_{nm})_{;k^\rmb}\right)^*
\quad\mathrm{for}\quad n\ne m
,
\end{align} 
with
Eq.~\eqref{eli.13}
\begin{align}\label{c-eli.13}
\gD_{nm}^{\rma}
=
v_{nn}^{\lda,\rma}-v_{mm}^{\lda,\rma}
,
\end{align}
and \eqref{na_rgendevn}
\begin{align}\label{c-na_rgendevn}
(r^{\rmb}_{nm})_{;k^{\rma}}
&=
-i\calt^{\rma\rmb}_{nm}
+
\frac{
r^{\rma}_{nm}
\Delta^{\rmb}_{mn}
+r^{\rmb}_{nm}
\Delta^{\rma}_{mn}
}
{\go^\lda_{nm}}
+
\frac{i}{\go^\lda_{nm}}
\sum_{\ell}
\bigg(
\go^\lda_{\ell m}
r^{\rma}_{n\ell}
r^{\rmb}_{\ell m}
-
\go^\lda_{n\ell}
r^{\rmb}_{n\ell}
r^{\rma}_{\ell m}
\bigg)
\nonumber\\
&\approx
\frac{
r^{\rma}_{nm}
\Delta^{\rmb}_{mn}
+r^{\rmb}_{nm}
\Delta^{\rma}_{mn}
}
{\go^\lda_{nm}}
+
\frac{i}{\go^\lda_{nm}}
\sum_{\ell}
\bigg(
\go^\lda_{\ell m}
r^{\rma}_{n\ell}
r^{\rmb}_{\ell m}
-
\go^\lda_{n\ell}
r^{\rmb}_{n\ell}
r^{\rma}_{\ell m}
\bigg)
\nonumber\\
(r^{\rmb}_{mn})_{;k^{\rma}}
&=
\left((r^{\rmb}_{nm})_{;k^{\rma}}\right)^*
,
\end{align}
where $\calt^{\rma\rmb}_{nm}\approx 0$.\\
$\bullet$ for $n=m$\\
Since 
$\calt^{\rma\rmb}_{nn}\approx (\hbar/m_e)\gd_{\rma\rmb}$,
Eq.~\eqref{a.3c} gives
\begin{align}\label{c-a.3c}
(v^{\lda,\rma}_{nn})_{;k^\rmb}
&=
-i\calt^{\rma\rmb}_{nn}
-
\sum_{\ell\ne n}
\go^\lda_{\ell n}
\bigg( 
r^{\rma}_{n\ell} 
r^\rmb_{\ell n}
+ 
r^\rmb_{n\ell} 
r^{\rma}_{\ell n}
\bigg)
\nonumber\\
&\approx
\frac{\hbar}{m_e}\gd_{\rma\rmb}
-
\sum_{\ell\ne n}
\go^\lda_{\ell n}
\bigg( 
r^{\rma}_{n\ell} 
r^\rmb_{\ell n}
+ 
r^\rmb_{n\ell} 
r^{\rma}_{\ell n}
\bigg)
.
\end{align} 
For Eq.~\eqref{c-a.2n} we need
\eqref{a.7}
\begin{align}\label{c-a.7}
 (\calc^\ell_{nm})_{;k^\rma}
&= 
i\sum_{q\ne nm}
\left(
r_{nq}^\rma
\calc^\ell_{qm}
-
\calc^\ell_{nq}
r_{qm}^\rma
\right)
+ir_{nm}^\rma(\calc^\ell_{mm}-\calc^\ell_{nn})
\nonumber\\
 (\calc^\ell_{mn})_{;\bfk}
&=
\big( (\calc^\ell_{nm})_{;\bfk}\big)^*
.
\end{align} 

For the scissor related term we have:
Eq.~\eqref{a.3b} , \eqref{choni.1} and \eqref{chon.2}
\begin{align}\label{c-a.3b}
\calv^{\cals,\rma,\ell}_{nm}
&=
\frac{1}{2}\sum_q\left( 
v^{\cals,\rma}_{nq}\calc^\ell_{qm}+\calc^\ell_{nq} v^{\cals,\rma}_{qm}
\right)
\nonumber\\
\left(\calv^{\cals,\rma}_{nm}\right)_{;k^\rmb}
&=
\frac{1}{2}\sum_q\left(
(v^{\cals,\rma}_{nq})_{;k^\rmb}\calc^\ell_{qm}
+  
v^{\cals,\rma}_{nq}(\calc^\ell_{qm})_{;k^\rmb}
+
(\calc^\ell_{nq})_{;k^\rmb} v^{\cals,\rma}_{qm}
+
\calc^\ell_{nq} (v^{\cals,\rma}_{qm})_{;k^\rmb}
\right)
,
\end{align}  
with Eqs.~\eqref{chon.2} and \eqref{choni.1}
\begin{align}\label{c-chon.2} 
v^{\cals,\rma}_{nm}=i\gS f_{mn}r^\rma_{nm}
,
\end{align}
\begin{align}\label{c-choni.1}
(v^{\cals,\rma}_{nm})_{;k^\rmb}=i\gS f_{mn}
(r^\rma_{nm})_{;k^\rmb}
,
\end{align}
where $\hbar\gS$ is the scissors correction.
Notice that
$v^{\cals,\rma}_{nn}=0$ and 
$(v^{\cals,\rma}_{nn})_{;k^\rmb}=0$.
Substuiting Eq.~\eqref{c-chon.2} into \eqref{c-a.3b}, we obtain
  \begin{align}\label{vs.cv}
      \mathcal{V}^{S,a,\ell}_{cv}
&= 
      \frac{i\Sigma}{2}
      \left[\sum_{v^{\prime}}r^{a}_{cv^{\prime}}C^{\ell}_{v^{\prime}v} 
          + \sum_{c^{\prime}}C^{\ell}_{cc^{\prime}}r^{a}_{c^{\prime}v}\right],
\nonumber\\
      \mathcal{V}^{S,a,\ell}_{vc}
&= 
      (\mathcal{V}^{S,a,\ell}_{cv})^*
    \end{align}
and    
  \begin{align}\label{vs.cc}
    \mathcal{V}^{S,a,\ell}_{cc} 
    &= -\Sigma\sum_{v}
    \text{Im}\left[r^{a}_{cv}C^{\ell}_{vc}\right],
  \end{align}
 \begin{align}\label{vs.vv}
    \mathcal{V}^{S,a,\ell}_{vv} 
    &= \Sigma\sum_{c}
    \text{Im}\left[r^{a}_{vc}C^{\ell}_{cv}\right],
  \end{align}
where the last two are real functions as they must, since they are velocities.
\subsection{Coding for $\calv^{\gs,\rma,\ell}_{nm}(\bfk)$} 
Recall that 
$\calv^{\lda,\rma,\ell}_{mn}=(\calv^{\lda,\rma,\ell}_{nm})^*$
and 
$\calv^{\cals,\rma,\ell}_{mn}=(\calv^{\cals,\rma,\ell}_{nm})^*$ 
\begin{itemize}
%%%
\item If \verb=-n= option is chosen in \verb=all_responses.sh=
\begin{itemize}
\item $\calv^{\lda,\rma,\ell}_{nm}$, comes from
  Eq.~\eqref{c-a.2}, coded in \verb=functions.f90=
\end{itemize}
%%%
\item If \verb=-n= option is NOT chosen in \verb=all_responses.sh=
\begin{itemize}
\item $\calv^{\lda,\rma,\ell}_{nm}$ 
is used from \verb=me_cpmn_*=
  which is Eq.~\eqref{eni.2} and is coded in \verb=sub_pmn_ascii.f90=
\end{itemize}
%%%
\end{itemize}
For either case
\begin{itemize}
\item $\calv^{\cals,\rma,\ell}_{nm}$ 
is obtained from
  Eqs.~\eqref{vs.cv}, \eqref{vs.cc} or \eqref{vs.vv}, depending on
  $nm$. This is coded in \verb=functions.f90= and used in 
\verb=set_input_ascii.f90=  
\end{itemize}
Thus,\\
$\bullet$ 
$\calv^{\gs,\rma,\ell}_{nm}(\bfk)=\calv^{\lda,\rma,\ell}_{nm}(\bfk)+\calv^{\cals,\rma,\ell}_{nm}(\bfk)$\\
is stored in \verb=calMomMatElem= 
array, constructed in 
\verb=set_input_ascii.f90=, and used in \verb=SRC_2latm= for
integrating the response function. A brave young soul, should change   
\verb=calMomMatElem= to \verb=calVelMatElem= in order to have a more
appropriate name. But as good old DNA, we construct upon available
ATGC; using the old structure, adding functionality and  keeping all
the usles non-codifying crap, thus making Darwin
 proud of us! 

\subsection{Coding for $(\calv^{\gs,\rma,\ell}_{nm}(\bfk))_{;k^\rmb}$}
\begin{itemize}
\item $\gD^\rma_{nm}$ available in array \verb=Delta=, 
calculated in \verb=set_input_ascii.f90=,
 and contains the
  contribution from $\bfv^\nl_{nm}(\bfk)$ if the \verb=-n= option is
  chosen in \verb=all_responses.sh= 
\item $(r^\rma_{nm}(\bfk))_{;k^\rmb}$
 available in array
  \verb=derMatElem=,
calculated in \verb=set_input_ascii.f90= and \verb=functions.f90=,
 and contains the
  contribution from $\bfv^\nl_{nm}(\bfk)$ if the \verb=-n= option is
  chosen in \verb=all_responses.sh= 
\item With above two we compute $(v^{\lda,\rma}_{nm}(\bfk))_{;k^\rmb}$ 
in \verb=set_input_ascii.f90=  and store it in \verb=gdVlda= for
diagonal and off diagonal terms.
\item $\big(\calc^\ell_{nm}(\bfk)\big)_{;k^\rma}$ is coded in 
in \verb=set_input_ascii.f90=  and store it in \verb=gdf= for
diagonal and off diagonal terms. Darwin at work!
\end{itemize}



\subsection{Bulk expressions}

For a bulk $\calc^\ell_{nm}(\bfk)=\gd_{nm}$, then
$(\calc^\ell_{nm}(\bfk))_{;\bfk}=0$, and Eq.~\eqref{c-a.1} reduces to
\begin{align}\label{choni.9}
v^{\gs,\rma}_{nm}
&=
v^{\lda,\rma}_{nm}
+
v^{\cals,\rma}_{nm}
\nonumber\\
\bfv^\gs_{nm}(\bfk)
&=
\left(1+\frac{\gS}{\go_c(\bfk)-\go_v(\bfk)}\right)\bfv^\lda_{nm}(\bfk)
\quad\quad n\notin D_m
\nonumber\\
\bfv^\gs_{nn}(\bfk)
&=
\bfv^\lda_{nn}(\bfk)
,
\end{align}
where in \verb=$TINIBA/latm= the values are coded in the array
called
\verb=momMatElem=.  
If option \verb=-n= is given while running
\verb=all_resposnses.sh=, then $\bfv^\nl_{nm}(\bfk)$ are included in 
\verb=momMatElem=. 
Also,
\begin{align}\label{c-a.1nn}
\left(
v^{\gs,\rma}_{nm}
\right)_{;k^\rmb}
&=
\left(
v^{\lda,\rma}_{nm}
\right)_{;k^\rmb}
+
\left(
v^{\cals,\rma}_{nm}
\right)_{;k^\rmb}
,
\end{align}
where with the r.h.s. expressions are given above. The generalized
derivative of $(\bfv^\gs_{nm})_{;\bfk}$ seems not to be needed  in any
response so far, therefore is not coded.
\subsection{Consistency check-up}

To check that the coding of 
$\calc^\ell_{nm}(\bfk)$ 
is correct, we can calculate $\calv^{\rma,\ell}_{nm}(\bfk)$ using
Eq.~\eqref{vcali} as follows
\begin{align}\label{ccu.1}
\calv^{\rma,\ell}_{nm}(\bfk)
&=
\frac{1}{2m_e}
\Big(
\calc^\ell(z)p^\rma
+
p^\rma\calc^\ell(z)
\Big)_{nm}
\nonumber\\
&=
\frac{1}{2m_e}
\sum_q
\Big(
\calc^\ell_{nq}p^\rma_{qm}
+
p^\rma_{nq}\calc^\ell_{qm}
\Big)
,
\end{align}
which must give the same results as those computed through
Eq.~\eqref{eni.2}.
Indeed, we have checked that this is the case. The
\verb=$TINIBA/util/consistency-of-cfmn.sh=
is used to check this.

\subsection{Subroutines}

The following subroutines/shells are involved in the coding,
and are documented between\\
\verb=#BMSd=\\
$\vdots$\\
\verb=#BMSu=\\
marks.
\begin{enumerate}
\item \verb=$TINIBA/utils/all_responses.sh=
\item \verb=$TINIBA/latm/SRC_1setinput/inparams.f90=\\
\textcolor{red}{Warning:} compile both\\
\verb=$TINIBA/latm/SRC_1setinput/= \\
and\\
\verb=$TINIBA/latm/SRC_2latm/= 
\item \verb=$TINIBA/latm/SRC_1setinput/set_input_ascii.f90=\\
\end{enumerate}
