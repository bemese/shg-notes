\documentclass[11pt]{article}
\usepackage{amsfonts}
\usepackage{amsmath}
\usepackage{parskip}
\usepackage{fullpage}
\usepackage{showkeys}
\usepackage[colorlinks,linkcolor={blue},citecolor={red}]{hyperref}
%\allowdisplaybreaks[1]%biutiful equation breaker!!!
%\usepackage{graphicx}
%\usepackage{ulem}
%\usepackage{subfigure}
%%%%%% defs
%%%% acronimos
\def\ps{\mathrm{ps}}
\def\lda{\mathrm{LDA}}
\def\rpa{\mathrm{RPA}}
\def\nl{\mathrm{nl}}
\def\acu{Accu-Check\textsuperscript{\textregistered}~Performa}
\def\goni{Glucometro \'Optico No Invasivo}
\def\Reg{\textsuperscript{\textregistered}}
\def\tiniba{ TINIBA\textsuperscript{\textregistered}}
\def\gw{{\it GW}}
\def\gsa{Generaci\'on del Segundo Arm\'onico}
\def\shg{Second Harmonic Generation}
\def\sfg{Sum Frequency Generation}
\def\sdf{Generaci\'on de Suma de Frecuencias}
%%%%% accent of i
\def\'#1{\if#1i{\accent19\i}\else{\accent19#1}\fi}
%%%%% compa\~nias
\def\micro{{\it Supermicro}}
\def\lufac{{\it LUFAC}}
%%%%% lugares
\def\lou{Laboratorio de \'Optica Ultrar\'apida}
\def\roma{Universidad de Roma II}
\def\tor{``Tor Vergata''}
\def\dti{Direcci\'on de Tecnolog\'{\i}a e Innovaci\'on}
\def\dfa{Direcci\'on de Formaci\'on Acd\'emica}
\def\dg{Direcci\'on General}
\def\da{Direcci\'on Administrativa}
\def\ifug{Insituto de F\'isica de la U. de Guanajuato}
\def\icf{Instituto de Ciencias F\'isicas}
\def\unam{Universidad Nacional Aut\'onoma de M\'exico}
\def\uguille{Universidad del Nordeste, Argentina}
\def\fotonica{Departamento de Fotonica}
\def\grupo{Propiedades \'Opticas de Nano-Sistemas, Interfases y Superficies}
\def\grupoa{PRONASIS}
%\def\grupo{Propiedades \'Opticas de Superficies e Interfases y Sistemas Nanosc\'opicos}
%\def\grupoa{POSISNA}
\def\di{Direcci\'on de Investigaci\'on}
\def\dfa{Direcci\'on de Formaci\'on Acad\'emica}
\def\cio{Centro de Investigaciones en \'Optica}
\def\ciod{Centro de Investigaciones en \'Optica, León, Guanajuato.}
\def\Conacyt{Consejo Nacional de Ciencia y Tecnolog\'ia}
\def\Concyteg{Consejo  de Ciencia y Tecnolog\'ia del Estado de Guanajuato}
\def\conacyt{CONACyT}
\def\concyteg{CONCyTEG}
\def\lagos{Centro Universitario de los Lagos}
\def\udeg{Universidad de Guadalajara}
\def\dinv{Direcci\'on de Investigaci\'on}
\def\dop{Department of Physics}
\def\uoft{University of Toronto}
\def\ua{University of Texas at Austin}
\def\icf{Instituto de Ciencias Físicas, UNAM, Cuernavaca}
%%%%% gente
%% grupo
\def\gabriel{Gabriel Ramos Ortíz}
\def\ramon{Ram\'on~ Carriles~ Jaimes}
\def\ramonm{Ram\acute{o}n~ Carriles~ Jaimes}
\def\enrique{Enrique~ Castro~ Camus}
\def\raul{Ra\'ul Alfonso V\'azquez Nava}
\def\raulm{Ra\acute{u}l~ Alfonso~ V\acute{a}zquez~ Nava}
\def\beto{Norberto~ Arzate~ Plata}
\def\bmsa{Bernardo S. Mendoza}
\def\bms{Bernardo~ Mendoza~ Santoyo}
%% alumnos
\def\cesar{C\'esar Castillo Quevedo}
\def\cabellos{Jos\'e Luis Cabellos Quiroz}
\def\tona{Tonatiuh Rangel Gordillo}
\def\temok{Juan Cuauhtemoc Salazar Gonz\'alez}
\def\adan{Luis Adan Mart\'inez Jim\'enez}
\def\sean{Sean Martin Anderson}
\def\reinaldo{Reinaldo Zapata Pe\~na}
%%% alumnos del grupo
%% enrique
%Maestria:
\def\jorgee{Jorge Alberto Caballero Mendoza}
\def\sofia{Sofía Carolina Corzo García}
\def\ruth{Ruth Julieta Medina López} 
%Doctorado: 
\def\juane{Juan Jes\'us S\'anchez S\'anchez}
%Licenciatura
\def\alma{Alma Gabriela González Patlán}
%(con Ramon): 
\def\sergioer{Sergio Augusto Romero Serv\'{\i}n}
%% Raul
%Maestria:
\def\enriquer{Enrique Arag\'on Navarro}%udg
\def\salomonr{Salom\'on Rodr\'{\i}guez Carrera}
\def\hectorr{H\'ector Santiago Hern\'andez}
\def\victor{Victor Manuel Villanueva Reyes}
%% Ramon
%Maestria:
\def\alfredora{Alfredo Campos Mej\'{\i}a}
%% Beto
%Doctorado
\def\noe{No\'e Gonz\'alez Baquedano}
%% otros
\def\liliana{Liliana Wilson Herr\'an}
\def\gerardo{Gerardo E. S\'anchez Garc\'{\i}a Rojas}
\def\amalia{Amalia Mart\'inez Garc\'{\i}a}
\def\nacho{Ing. José Ignacio Diego Manrique}
\def\tere{Teresita del Niño Jesús Pérez Hernández}
\def\elder{Elder de la Rosa Cruz}
\def\gonzalo{Gonzalo P\'aez Padilla}
\def\wlm{W. Luis Moch\'an Backal}
\def\oracio{Oracio C. Barbosa Garc\'ia}
\def\hector{H\'ector Hugo S\'anchez Hern\'andez}
\def\marco{Marco Antonio Escobar-Acevedo}
\def\gil{Alejandro Gil-Villegas Montiel}
\def\ernesto{Ernesto Carlos Cort\'es Morales}
\def\fms{Fernando Mendoza Santoyo}
\def\cuevas{Francisco Javier Cuevas de la Rosa}
\def\brenda{Brenda Esmeralda Matr\'inez Z\'erega}
\def\guille{Guillermo Ortiz}
\def\cesar{Cesar Castillo Quevedo}
\def\sipe{Prof. John Sipe}
\def\mike{Prof. Michael Downer}
\def\jems{Jorge Enrique Mej\'ia S\'anchez}
\def\lamon{Ram\'on Rodr\'iguez Vera}
\def\ldp{Luis de la Pe\~na}
\def\sole{Rodolfo Del Sole}
\def\lucia{Lucia Reining}
\def\sch{Schr\"odinger}
\def\Cuevas{Francisco J. Cuevas de la Rosa}
%%%%% categorias
\def\ita{Investigador Titular A}
\def\itb{Investigador Titular B}
\def\itc{Investigador Titular C}
\def\itd{Investigador Titular D}
\def\ite{Investigador Titular E}
\def\sr{Senior Researcher}
\def\iac{Investigador Asociado C}    
\def\alm{Alumno de Maestr\'ia}
\def\ald{Alumno de Doctorado}
\def\all{Alumno de Licenciatura}
\def\adei{Asistente de Investigaci\'on}
\def\sniIII{S.N.I. nivel III}
\def\sni{S.N.I.}
\def\cv{Currículum Vitae}
%%%%%% fonts
\def\tit{\sf}
\def\col{\sc}
\def\alu{\it} % for students
\def\cual{2$^{do}$}
\def\anno{2005}
\def\spe{\vspace{.12cm}}
%%%%%% cosas
\def\capa{capa-a-capa}
\def\espin{espintr\'onica}
\def\oespin{optoespintr\'onica}
\def\proyecto{Photon Assisted Spintronics}
\def\npro{48915}
\def\cvk{cv\mathbf{k}}
\def\cvkp{c'v'\mathbf{k}'}
%%%%%% revistas
\def\prb{Physical Review B}
\def\prl{Physical Review Letters}
\def\ol{Optics Letters}
\def\opn{Optics and Photonics News}
\def\pssc{physica status solidi (c)}
%%%%%%%%%%%%%%%%%%%%%%%%%%%%%%%%%%%%%%%
%%%%%% griegas
\def\ga{\alpha}
\def\gb{\beta}
\def\gga{\gamma}
\def\gGa{\Gamma}
\def\go{\omega}
\def\got{\tilde\omega}
\def\gO{\Omega}
\def\gr{{\rho}}
\def\ge{\epsilon}
\def\ve{\varepsilon}
\def\gve{\varepsilon}
\def\gd{\delta}
\def\gD{\Delta}
\def\gl{\lambda}
\def\gs{\sigma}
\def\gS{\Sigma}
\def\gbs{\overline{\sigma}}
%%%%%% griegas with tilde
\def\gta{\tilde{\alpha}}
\def\gtb{\tilde{\beta}}
\def\gtga{\tilde{\gamma}}
\def\gto{\tilde{\omega}}
\def\gtO{\tilde{\Omega}}
\def\gtr{\tilde{\rho}}
\def\gte{\tilde{\epsilon}}
\def\vte{\tilde{\varepsilon}}
\def\gtd{\tilde{\delta}}
\def\gtD{\tilde{\Delta}}
\def\gtl{\tilde{\lambda}}
\def\gts{\tilde{\sigma}}
\def\gtS{\tilde{\Sigma}}
%%%%%% romans with tilde
\def\bftr{\tilde{\mathbf{r}}}
\def\bftp{\tilde{\mathbf{p}}}
\def\bftv{\tilde{\mathbf{v}}}
\def\ta{\tilde{a}}
\def\tb{\tilde{b}}
\def\tr{\tilde{r}}
\def\tp{\tilde{p}}
\def\tV{\tilde{V}}
\def\tv{\tilde{v}}
%%
\newcommand{\ham}{\hat{\mathcal H}}
%%%%%% bra kets
\newcommand{\la}{\langle}
\newcommand{\ra}{\rangle}
\newcommand{\ket}[1]{| #1 \rangle}
\newcommand{\bra}[1]{\langle #1 |}
\newcommand{\braket}[2]{\langle {#1} | {#2} \rangle}
\newcommand{\ketbra}[2]{| {#1} \rangle {#1} \langle {#2} |}
\newcommand{\ave}[1]{\langle {#1} \rangle}
\newcommand{\dotp}[2]{\mathbf{#1} \cdot \mathbf{#2}}
%%%%%% averages
\newcommand{\prom}[1]{\langle {#1} \rangle}
%%%%%% creation and annihilation operators
\newcommand{\oa}{\hat a^{\tiny\strut}}
\newcommand{\oad}{\hat a^\dagger}
\newcommand{\oadk}{\hat a^\dagger_{\mathbf k}}
\newcommand{\oak}{\hat a^{\tiny\strut}_{\mathbf k}}
\newcommand{\obd}[1]{\hat b^\dagger_{#1}}
\newcommand{\ob}[1]{\hat b^{\tiny\strut}_{#1}}
%%%%%% Caligraphic
\newcommand{\cala}{{\mathbf{\cal A}}}
\newcommand{\calb}{{\mathbf{\cal B}}}
\newcommand{\calc}{{\mathbf{\cal C}}}
\newcommand{\cald}{{\mathbf{\cal D}}}
\newcommand{\cale}{{\mathbf{\cal E}}}
\newcommand{\calf}{{\mathbf{\cal F}}}
\newcommand{\calp}{{\mathbf{\cal P}}}
\newcommand{\calg}{{\mathbf{\cal G}}}
\newcommand{\calv}{{\mathbf{\cal V}}}
\newcommand{\calo}{{\cal O}}
\newcommand{\calr}{{\cal R}}
\newcommand{\cals}{{\cal S}}
\newcommand{\calw}{{\cal W}}
\newcommand{\calbd}{\boldsymbol{\mathcal{\cal D}}}
\newcommand{\calbp}{\boldsymbol{\mathcal{\cal P}}}
\newcommand{\calbv}{\boldsymbol{\mathcal{\cal V}}}
\newcommand{\calbs}{\boldsymbol{\mathcal{\cal S}}}
%%%%%% mathematicla bold roman & greek
\newcommand{\mbf}[1]{\mathbf{#1}}
\newcommand{\mbg}[1]{\boldsymbol{\mathcal {#1}}}
\newcommand{\bfA}{\mathbf{A}}
\newcommand{\bfB}{\mathbf{B}}
\newcommand{\bfC}{\mathbf{C}}
\newcommand{\bfD}{\mathbf{D}}
\newcommand{\bfE}{\mathbf{E}}
\newcommand{\bfF}{\mathbf{F}}
\newcommand{\bfG}{\mathbf{G}}
\newcommand{\bfH}{\mathbf{H}}
\newcommand{\bfI}{\mathbf{I}}
\newcommand{\bfJ}{\mathbf{J}}
\newcommand{\bfK}{\mathbf{K}}
\newcommand{\bfL}{\mathbf{L}}
\newcommand{\bfM}{\mathbf{M}}
\newcommand{\bfN}{\mathbf{N}}
\newcommand{\bfP}{\mathbf{P}}
\newcommand{\bfR}{\mathbf{R}}
\newcommand{\bfS}{\mathbf{S}}
\newcommand{\bfT}{\mathbf{T}}
\newcommand{\bfU}{\mathbf{U}}
\newcommand{\bfV}{\mathbf{V}}
\newcommand{\bfW}{\mathbf{W}}
\newcommand{\bfX}{\mathbf{X}}
\newcommand{\bfY}{\mathbf{Y}}
\newcommand{\bfZ}{\mathbf{Z}}
\newcommand{\bfa}{\mathbf{a}}
\newcommand{\bfb}{\mathbf{b}}
\newcommand{\bfc}{\mathbf{c}}
\newcommand{\bfd}{\mathbf{d}}
\newcommand{\bfe}{\mathbf{e}}
\newcommand{\bff}{\mathbf{f}}
\newcommand{\bfg}{\mathbf{g}}
\newcommand{\bfh}{\mathbf{h}}
\newcommand{\bfi}{\mathbf{i}}
\newcommand{\bfj}{\mathbf{j}}
\newcommand{\bfk}{\mathbf{k}}
\newcommand{\bfn}{\mathbf{n}}
\newcommand{\bfp}{\mathbf{p}}
\newcommand{\bfq}{\mathbf{q}}
\newcommand{\bfr}{\mathbf{r}}
\newcommand{\bfs}{\mathbf{s}}
\newcommand{\bft}{\mathbf{t}}
\newcommand{\bfu}{\mathbf{u}}
\newcommand{\bfv}{\mathbf{v}}
\newcommand{\bfx}{\mathbf{x}}
\newcommand{\bfy}{\mathbf{y}}
\newcommand{\bfz}{\mathbf{z}}
\newcommand{\bfzero}{\mathbf{0}}
\newcommand{\bfone}{\mathbf{1}}
%
\newcommand{\bfgeta}{\boldsymbol{\eta}}
\newcommand{\bfSig}{\boldsymbol{\Sigma}}
\newcommand{\bfsig}{\boldsymbol{\sigma}}
\newcommand{\bfgS}{\boldsymbol{\Sigma}}
\newcommand{\bfgs}{\boldsymbol{\sigma}}
\newcommand{\bfga}{\boldsymbol{\alpha}}
\newcommand{\bfgb}{\boldsymbol{\beta}}
\newcommand{\bfge}{\boldsymbol{\epsilon}}
\newcommand{\bfgvare}{\boldsymbol{\varepsilon}}
\newcommand{\bfgg}{\boldsymbol{\gamma}}
\newcommand{\bfgG}{\boldsymbol{\Gamma}}
\newcommand{\bfgphi}{\boldsymbol{\phi}}
\newcommand{\bfgpsi}{\boldsymbol{\psi}}
\newcommand{\bfgD}{\boldsymbol{\Delta}}
\newcommand{\bfgPhi}{\boldsymbol{\Phi}}
\newcommand{\bfgPsi}{\boldsymbol{\Psi}}
\newcommand{\bfgxi}{\boldsymbol{\xi}}
\newcommand{\bfgchi}{\boldsymbol{\chi}}
\newcommand{\bfgnabla}{\boldsymbol{\nabla}}
\newcommand{\bfgnu}{\boldsymbol{\nu}}
\newcommand{\bfgmu}{\boldsymbol{\mu}}
\newcommand{\bfgrho}{\boldsymbol{\rho}}
\newcommand{\bfgRho}{\boldsymbol{\Rho}}
%%%%%% nabla
\newcommand{\nablak}{\frac{\partial}{\partial\mathbf{k}} }
%%%%%% ; derivative
\def\gk{{;\mathbf k}}
%%%%%% k derivative
\newcommand{\deriv}[2] {\frac{\partial {#1}} {\partial {#2} }}
%%%%%% prime for \sum
\def\prima{\strut^{_{'}}}
%%%%%% subindices
%\def\eti{n\bfk}
\newcommand{\eti}[1]{_{#1 \bfk}}
\newcommand{\etiup}[1]{_{#1 \bfk s}}
\newcommand{\etidn}[1]{_{#1 \bfk \bar{s}}}
%%%%% superindice to push down the subindex in greeks!
\def\pd{^{\strut}}
%%%%% gauges
\def\rde{$\bfr\cdot\bfE$~}
\def\rder{length-gauge}
\def\pda{$\bfp\cdot\bfA$~}
\def\vda{$\bfv\cdot\bfA$~}
\def\vdar{velocity-gauge}
%%%%% integral over k
\def\intk{\int\frac{d^3k}{8\pi^3}}
%%%%% roman indices
\def\rmi{\mathrm{i}}
\def\rmj{\mathrm{j}}
\def\rmk{\mathrm{k}}
\def\rml{\mathrm{l}}
\def\rmr{\mathrm{r}}
\def\rma{\mathrm{a}}
\def\rmb{\mathrm{b}}
\def\rmc{\mathrm{c}}
\def\rmd{\mathrm{d}}
\def\rme{\mathrm{e}}
\def\rmv{\mathrm{v}}
\def\rmz{\mathrm{z}}
\def\rmx{\mathrm{x}}
\def\rmy{\mathrm{y}}
\def\rmH{\mathrm{H}}
\def\rmG{\mathrm{G}}
\def\rmW{\mathrm{W}}
%%%%% functions
\def\erf{\mathrm{erf}}
\def\erfc{\mathrm{erfc}}
\def\erfi{\mathrm{erfi}}

%iave: C01243171 y 0124317111

%\usepackage{bm}

\begin{document}

We start with the expression for the susceptibility for the intraband transtitions,

\begin{equation}\label{chii}
\chi_{i,\text{a}\text{b}\text{c}}^{s,\ell}=-\frac{e^3}{\Omega\hbar^2\omega_3}\sum_{mn\mathbf{k}}\frac{\mathcal{V}_{mn}^{\Sigma,\text{a},\ell}}{\omega^S_{nm}-\omega_3}\left(\frac{f_{mn}r_{nm}^{\text{b}}}{\omega^S_{nm}-\omega_\beta}\right)_{;k^{\text{c}}},
\end{equation} 

where \emph{s} denotes \emph{surface} and \emph{S} refers to the \emph{scissors} correction. This expression diverges as $\omega_{3} \rightarrow 0$. To eliminate this divergence we take the partial fraction expansion,

\begin{align}\label{pfi} 
I &= C \left[-\frac{1}{2(\omega^{S}_{nm})^{2}}\frac{1}{\omega^{S}_{nm}-\omega}+\frac{2}{(\omega^{S}_{nm})^{2}}\frac{1}{\omega^{S}_{nm}-2\omega}+\frac{1}{2(\omega^{S}_{nm})^{2}}\frac{1}{\omega}\right]\nonumber\\
&- D \left[-\frac{3}{2(\omega^{S}_{nm})^{2}}\frac{1}{\omega^{S}_{nm}-\omega}+\frac{4}{(\omega^{S}_{nm})^{3}}\frac{1}{\omega^{S}_{nm}-2\omega}+\frac{1}{2(\omega^{S}_{nm})^{3}}\frac{1}{\omega}-\frac{1}{2(\omega^{S}_{nm})^{2}}\frac{1}{(\omega^{S}_{nm}-\omega)^2}\right],
\end{align} 

where $C = f_{mn}\mathcal{V}^{\Sigma,\text{a}}_{mn}(r^{\text{LDA},\text{b}}_{nm})_{;k^{\text{c}}}$, and $D=f_{mn}\mathcal{V}^{\Sigma,\text{a}}_{mn}r^{\text{b}}_{nm}\Delta^{\text{c}}_{nm}$.

Time-reversal symmetry leads to the following relationships:

\begin{align}\label{time_reversal}
\mathbf{r}_{mn}(\mathbf{k})|_{-\mathbf{k}}                       &=  \mathbf{r}_{nm}(\mathbf{k})|_{\mathbf{k}},                                 \nonumber\\
(\mathbf{r}_{mn})_{;\mathbf{k}}(\mathbf{k})|_{-\mathbf{k}}       &=  (-\mathbf{r}_{nm})_{;\mathbf{k}}(\mathbf{k})|_{\mathbf{k}},                \nonumber\\
\mathcal{V}^{\Sigma,\text{a}}_{mn}(\mathbf{k})|_{-\mathbf{k}}    &=  -\mathbf{\mathcal{V}}_{nm}^{\Sigma,\text{a}}(\mathbf{k})|_{\mathbf{k}},             \\
\omega_{mn}^{S}(\mathbf{k})|_{-\mathbf{k}}                       &=  \omega_{mn}^{S}(\mathbf{k})|_{\mathbf{k}},                                 \nonumber\\
\Delta^a_{nm}(\mathbf{k})|_{-\mathbf{k}}                         &=  -\Delta^a_{nm}(\mathbf{k})|_{\mathbf{k}}.                                  \nonumber
\end{align}

For a clean cold semiconductor, $f_{n} = 1$ for an occupied or valence $(n = v)$ band, and $f_{n} = 0$ for an empty or conduction $(n = c)$ band independent of $\mathbf{k}$, and $f_{nm}=-f_{mn}$.

The $\frac{1}{\omega}$ terms cancel each other out. We notice that the energy denominators are invariant under $\mathbf{k} \rightarrow - \mathbf{k}$, and then we only look at the numerators, then

\begin{align}\label{ct}
C \rightarrow f_{mn}\mathcal{V}^{\Sigma,\text{a}}_{mn}\left(r^{\text{LDA},\text{b}}_{nm}\right)_{;k^{\text{c}}}|_{\mathbf{k}}
&+ f_{mn}\mathcal{V}^{\Sigma,\text{a}}_{mn}\left(r^{\text{LDA},\text{b}}_{nm}\right)_{;k^{\text{c}}}|_{-\mathbf{k}}\nonumber\\
&= f_{mn}\left[\mathcal{V}^{\Sigma,\text{a}}_{mn}\left(r^{\text{LDA},\text{b}}_{nm}\right)_{;k^{\text{c}}}|_{\mathbf{k}} + \left(-\mathcal{V}^{\Sigma,\text{a}}_{nm}\right)\left(-r^{\text{LDA},\text{b}}_{mn}\right)_{;k^{\text{c}}}|_{\mathbf{k}}\right]\nonumber\\
&= f_{mn}\left[\mathcal{V}^{\Sigma,\text{a}}_{mn}\left(r^{\text{LDA},\text{b}}_{nm}\right)_{;k^{\text{c}}} + \mathcal{V}^{\Sigma,\text{a}}_{nm}\left(r^{\text{LDA},\text{b}}_{mn}\right)_{;k^{\text{c}}}\right]\nonumber\\
&= f_{mn}\left[\mathcal{V}^{\Sigma,\text{a}}_{mn}\left(r^{\text{LDA},\text{b}}_{nm}\right)_{;k^{\text{c}}} + \left(\mathcal{V}^{\Sigma,\text{a}}_{mn}\left(r^{\text{LDA},\text{b}}_{nm}\right)_{;k^{\text{c}}}\right)^*\right]\nonumber\\
&= 2f_{mn}\,\mathrm{Re}\left[\mathcal{V}^{\Sigma,\text{a}}_{mn}\left(r^{\text{LDA},\text{b}}_{nm}\right)_{;k^{\text{c}}}\right],
\end{align}

and likewise,

\begin{align}\label{dt}
D \rightarrow f_{mn}\mathcal{V}^{\Sigma,\text{a}}_{mn}r^{\text{b}}_{nm}\Delta^{\text{c}}_{nm}|_{\mathbf{k}} 
&+ f_{mn}\mathcal{V}^{\Sigma,\text{a}}_{mn}r^{\text{b}}_{nm}\Delta^{\text{c}}_{nm}|_{-\mathbf{k}}\nonumber\\
&= f_{mn}\left[\mathcal{V}^{\Sigma,\text{a}}_{mn}r^{\text{b}}_{nm}\Delta^{\text{c}}_{nm}|_{\mathbf{k}} + \left(-\mathcal{V}^{\Sigma,\text{a}}_{nm}\right)r^{\text{b}}_{mn}\left(-\Delta^{\text{c}}_{nm}\right)|_{\mathbf{k}}\right]\nonumber\\
&= f_{mn}\left[\mathcal{V}^{\Sigma,\text{a}}_{mn}r^{\text{b}}_{nm} + \mathcal{V}^{\Sigma,\text{a}}_{nm}r^{\text{b}}_{mn}\right]\Delta^{\text{c}}_{nm}\nonumber\\
&= f_{mn}\left[\mathcal{V}^{\Sigma,\text{a}}_{mn}r^{\text{b}}_{nm} + \left(\mathcal{V}^{\Sigma,\text{a}}_{mn}r^{\text{b}}_{nm}\right)^*\right]\Delta^{\text{c}}_{nm}\nonumber\\
&= 2f_{mn}\,\mathrm{Re}\left[\mathcal{V}^{\Sigma,\text{a}}_{mn}r^{\text{b}}_{nm}\right]\Delta^{\text{c}}_{nm}.
\end{align}

The last term in the second line of \eqref{pfi} is dealt with as follows,

\begin{align}\label{dresn}
\frac{D}{2(\omega^S_{nm})^2}\frac{1}{(\omega^S_{nm}-\omega)^2} 
&= \frac{f_{mn}}{2}\frac{\mathcal{V}^{\Sigma,\text{a}}_{mn}r^{\text{b}}_{nm}}{(\omega^S_{nm})^2}\frac{\Delta^{\text{c}}_{nm}}{(\omega^S_{nm}-\omega)^2} = \frac{f_{mn}}{2}\frac{\mathcal{V}^{\Sigma,\text{a}}_{mn}r^{\text{b}}_{nm}}{(\omega^S_{nm})^2}\left(\frac{1}{\omega^S_{nm}-\omega}\right)_{;k^{\text{c}}}\nonumber\\
&= -\frac{f_{mn}}{2}\left(\frac{\mathcal{V}^{\Sigma,\text{a}}_{mn}r^{\text{b}}_{nm}}{(\omega^S_{nm})^2}\right)_{;k^{\text{c}}}\frac{1}{\omega^S_{nm}-\omega}.
\end{align} 

We use the fact that

\begin{equation}\label{wk}
(\omega^S_{nm})_{;k^{\text{c}}}=(\omega^\text{LDA}_{nm})_{;k^{\text{c}}} = \frac{p_{nn}^{\text{c}}-p_{mm}^{\text{c}}}{m_{e}} \equiv \Delta_{nm}^{\text{c}},
\end{equation}

and for the last line, we performed an integration by parts over the Brillouin zone, where the contribution from the edges vanishes. Using the chain rule, we obtain

\begin{equation}\label{chr}
\left(\frac{\mathcal{V}^{\Sigma,\text{a}}_{mn}r^{\text{b}}_{nm}}{(\omega^{S}_{nm})^2}\right)_{;k^{\text{c}}} = \frac{r^{\text{b}}_{nm}}{(\omega^{S}_{nm})^2}\left(\mathcal{V} ^{\Sigma,\text{a}}_{mn}\right)_{;k^{\text{c}}} + \frac{\mathcal{V}^{\Sigma,\text{a}}_{mn}}{(\omega^{S}_{nm})^2}\left(r^{\text{b}}_{nm}\right)_{;k^{\text{c}}} - \frac{\mathcal{V}^{\Sigma,\text{a}}_{mn}r^{\text{b}}_{nm}}{2(\omega^{S}_{nm})^3}\left(
\omega^{S}_{nm}\right)_{;k^{\text{c}}}.
\end{equation}

%% verify that the generalized derivative actually behaves this way. you might need to look into older versions when we were still using calR.
We will check each term of \eqref{chr} over $\mathbf{k} \rightarrow - \mathbf{k}$. The first term is reduced to

\begin{align}\label{first_term_gen_deriv}
\frac{r^{\text{b}}_{nm}}{(\omega^{S}_{nm})^{2}}\left(\mathcal{V}^{\Sigma,\text{a}}_{mn}\right)_{;k^{\text{c}}}|_{\mathbf{k}} + \frac{r^{\text{b}}_{nm}}{(\omega^{S}_{nm})^{2}}\left(\mathcal{V}^{\Sigma,\text{a}}_{mn}\right)_{;k^{\text{c}}}|_{-\mathbf{k}}
&= \frac{r^{\text{b}}_{nm}}{(\omega^{S}_{nm})^{2}}\left(\mathcal{V}^{\Sigma,\text{a}}_{mn}\right)_{;k^{\text{c}}}|_{\mathbf{k}} - \frac{r^{\text{b}}_{mn}}{(\omega^{S}_{nm})^{2}}\left(\mathcal{V}^{\Sigma,\text{a}}_{nm}\right)_{;k^{\text{c}}}|_{\mathbf{k}}\nonumber\\
&= \frac{1}{(\omega^{S}_{nm})^{2}}\left[r^{\text{b}}_{nm}\left(\mathcal{V}^{\Sigma,\text{a}}_{mn}\right)_{;k^{\text{c}}} - \left(r^{\text{b}}_{nm}\left(\mathcal{V}^{\Sigma,\text{a}}_{mn}\right)_{;k^{\text{c}}}\right)^*\right]\nonumber\\
&= \frac{2i}{(\omega^{S}_{nm})^{2}}\mathrm{Im}\left[r^{\text{b}}_{nm}\left(\mathcal{V}^{\Sigma,\text{a}}_{mn}\right)_{;k^{\text{c}}}\right],
\end{align}

the second term is reduced to

\begin{align}\label{second_term_gen_deriv}
\frac{\mathcal{V}^{\Sigma,\text{a}}_{mn}}{(\omega^{S}_{nm})^{2}}\left(r^{\text{b}}_{nm}\right)_{;k^{\text{c}}}|_{\mathbf{k}} + \frac{\mathcal{V}^{\Sigma,\text{a}}_{mn}}{(\omega^{S}_{nm})^{2}}\left(r^{\text{b}}_{nm}\right)_{;k^{\text{c}}}|_{-\mathbf{k}}
&= \frac{\mathcal{V}^{\Sigma,\text{a}}_{mn}}{(\omega^{S}_{nm})^{2}}\left(r^{\text{b}}_{nm}\right)_{;k^{\text{c}}}|_{\mathbf{k}} + \frac{\mathcal{V}^{\Sigma,\text{a}}_{nm}}{(\omega^{S}_{nm})^{2}}\left(r^{\text{b}}_{mn}\right)_{;k^{\text{c}}}|_{\mathbf{k}}\nonumber\\
&= \frac{1}{(\omega^{S}_{nm})^{2}}\left[\mathcal{V}^{\Sigma,\text{a}}_{mn}\left(r^{\text{b}}_{nm}\right)_{;k^{\text{c}}} + \left(\mathcal{V}^{\Sigma,\text{a}}_{mn}\left(r^{\text{b}}_{nm}\right)_{;k^{\text{c}}}\right)^*\right]\nonumber\\
&= \frac{2}{(\omega^{S}_{nm})^{2}}\mathrm{Re}\left[\mathcal{V}^{\Sigma,\text{a}}_{mn}\left(r^{\text{b}}_{nm}\right)_{;k^{\text{c}}}\right],
\end{align}

%% you also need to verify the properties for the generalized derivative of omega from appendix D.
and the third term is reduced to

\begin{align}\label{third_term_gen_deriv}
\frac{\mathcal{V}^{\Sigma,\text{a}}_{mn}r^{\text{b}}_{nm}}{2(\omega^{S}_{nm})^{3}}\left(\omega^{S}_{nm}\right)_{;k^{\text{c}}}|_{\mathbf{k}} + \frac{\mathcal{V}^{\Sigma,\text{a}}_{mn}r^{\text{b}}_{nm}}{2(\omega^{S}_{nm})^{3}}\left(\omega^{S}_{nm}\right)_{;k^{\text{c}}}|_{-\mathbf{k}}
&= \frac{\mathcal{V}^{\Sigma,\text{a}}_{mn}r^{\text{b}}_{nm}}{2(\omega^{S}_{nm})^{3}}\left(\omega^{S}_{nm}\right)_{;k^{\text{c}}}|_{\mathbf{k}} - \frac{\mathcal{V}^{\Sigma,\text{a}}_{nm}r^{\text{b}}_{mn}}{2(\omega^{S}_{nm})^{3}}\left(\omega^{S}_{nm}\right)_{;k^{\text{c}}}|_{\mathbf{k}}\nonumber\\
%&= \frac{1}{(\omega^{S}_{nm})^{2}}\left[r^{\text{b}}_{nm}\left(\mathcal{V}^{\Sigma,\text{a}}_{mn}\right)_{;k^{\text{c}}} - \left(r^{\text{b}}_{nm}\left(\mathcal{V}^{\Sigma,\text{a}}_{mn}\right)_{;k^{\text{c}}}\right)^*\right]\nonumber\\
%&= \frac{2i}{(\omega^{S}_{nm})^{2}}\mathrm{Im}\left[r^{\text{b}}_{nm}\left(\mathcal{V}^{\Sigma,\text{a}}_{mn}\right)_{;k^{\text{c}}}\right].
\end{align}

\end{document}
