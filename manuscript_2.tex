\documentclass[floatfix,prb,aps,superscriptaddress,11pt]{revtex4}
\usepackage{amsfonts}
\usepackage{amsmath}
\usepackage{graphicx}
%\usepackage{showkeys}
\usepackage{ulem}
\usepackage{bm}
\setcounter{MaxMatrixCols}{10}
\begin{document}
\title{$\mathbf{v}\cdot\mathbf{A}$~ versus $\mathbf{r}\cdot\mathbf{E}$~ calculations of second harmonic generation for semiconductors}
\author{J.L. Cabellos}
\author{Bernardo S. Mendoza}
\email[email:]{bms@cio.mx}
\author{M.A. Escobar}
\affiliation{Department of Photonics, Centro de Investigaciones en \'Optica,\\
Le\'on, Guanajuato, M\'exico}
\author{F. Nastos}
\author{J.E. Sipe}
\affiliation{Department of Physics and Institute for Optical Sciences, University of
Toronto,\\
60 St. George St., Toronto, Ontario, Canada {M5S-1A7}}
\date{\today}

\begin{abstract}
Calculations 
of the second harmonic susceptibility tensor
$\chi^{abc}(-2\omega;\omega,\omega)$
are presented for 
bulk semiconductors within both the $\mathbf{v}\cdot\mathbf{A}$~ 
and the $\mathbf{r}\cdot\mathbf{E}$~ gauge. 
The description of the semiconductor states incorporates the ``scissors'' 
Hamiltonian commonly used to obtain the correct band gap.
The non-locality of the scissors correction leads to                               
new terms in $\chi^{abc}(-2\omega;\omega,\omega)$
not considered before within a sum-over-states                                    
approach to the $\mathbf{v}\cdot\mathbf{A}$~ gauge.
Using this new expression, we show that the results of the two gauges give 
the same result for $\chi^{abc}(-2\omega;\omega,\omega)$, within very
good numerical accuracy.
As part of the derivation, 
we clarify the well known result for the linear optical 
response which states that the scissors
correction rigidly shifts the spectrum along the energy axis, keeping the
line-shape intact. The calculation is presented for GaAs using an \textit{
all-electron} and a \textit{pseudopotential} scheme.
\end{abstract}

\pacs{42.65.An,42.65.Ky,42.70.Nq,78.20.e,71.15.Mb}
\keywords{gauge, SHG }
\maketitle

%%%%%%%%%%%%%%%%%

\section{Introduction}

The development of new nonlinear optical materials is an active area of
research, with works ranging from the study and growth of new nonlinear
crystals to the design of novel metamaterials. Perhaps the simplest
nonlinear process, second harmonic generation (SHG), is one of the most
important for the generation of new frequencies, as a spectroscopic probe,
and because its reverse process -- spontaneous parametric downconversion,
which is described by the same nonlinear susceptibility -- can be used to
generate entangled photons for application in quantum information
processing.

The numerical calculation of any nonlinear optical response is a nontrivial
task, and different methodologies and numerical approaches have been
employed.
Our interest is in strategies that can be applied to study the nonlinear optical 
response of a material over a wide frequency range, and in a regime where a 
perturbative treatment is appropriate. The first attempt along these lines is 
the work of Butcher and McLean,\cite{butcherPPS63} where the resulting equations
appeared to be plagued by divergences that appear in the DC (static) limit.
Aspnes\cite{aspnesPRB72} showed that in the static limit these divergences
are only apparent, in that the coefficients that multiply the
divergent terms vanish;
but his proof was limited to cubic crystals. Ghahramani,
Moss and Sipe,\cite{ghahramaniPRB91} gave a more general proof of the
disappearance of these apparent divergence for cold ($T=0$ K), undoped
semiconductors of any crystal class. 
Levine,\cite{levinePRB94} presented a formula for the  
nonlinear
second order susceptibility tensor
where the scissors approximation is properly introduced, but 
the expressions are difficult to compute.
These studies used what is sometimes
called the ``velocity gauge'' or ``$\mathbf{v}\cdot\mathbf{A}$~ gauge'' for the treatment of the
coupling of an electron to the electromagnetic field, where $\mathbf{v}$ is the
velocity operator of the electron and $\mathbf{A}$ is the vector potential
specifying the electromagnetic field. 
Later, Aversa and Sipe\cite{aversaPRB95} showed that a                                  
divergence free expression for the nonlinear second order                                           
susceptibility tensor
$\chi^{abc}(-2\omega;\omega,\omega)$
could be more easily obtained using what is                               
sometimes called the ``length gauge'' or ``$\mathbf{r}\cdot\mathbf{E}$~
formulation''. Here $\mathbf{r}$
 is the position operator and $\mathbf{E}$ is the electric field.

In the works of Rashkeev et al.,\cite{rashkeevPRB98} and Hughes and
Sipe,\cite{hughesPRB96} the length-gauge formulation
was used to evaluate $\chi^{abc}(-2\omega;\omega,\omega)$ for
several zinc-blende semiconductors
within an \textit{ab initio} scheme.
The more recent work of Leitsmann et al.,
\cite{leitsmannPRB05} extends the velocity-gauge~approach to include excitonic and
local field interactions in GaAs. Quasi-particle effects, at the scissors
correction level, have been correctly incorporated by Nastos et al.\cite
{nastosPRB05} in the length-gauge approach, and before this 
Adolph and Bechstedt\cite{adolphPRB98} discussed how to include these
effects, even beyond the
scissors approximation, within the velocity-gauge approach. Surface second harmonic
generation has also been studied within the velocity-gauge~scheme with good success,
\cite{mendozaPRB01,mendozaPRL98,reiningPRB94} and $\chi^{abc}(-2\omega;\omega,\omega)$ spectra have
been calculated for superlattices within both 
the length-gauge~approach\cite{sharmaPRB03} and the 
velocity-gauge~approach.\cite{ghahramaniPRB91}

However, a full comparison between calculations using these two different
approaches has not been done. One goal of this article is to establish the
equivalence between the length-gauge~and velocity-gauge~schemes. Of course, it is well known
that measurable quantities must be gauge invariant, and indeed we show in
this article that the expressions for $\chi^{abc}(-2\omega;\omega,\omega)$ from the two different
approaches give the same result. In order to do so, we derive a new
expression for $\chi^{abc}(-2\omega;\omega,\omega)$ within the velocity-gauge~that properly takes into
account the non-local nature of the scissors Hamiltonian. In all previous
calculations of $\chi^{abc}(-2\omega;\omega,\omega)$ within the velocity-gauge, the scissors implementation
was carried out by following its implementation for linear optical response.
We show that this na\"{\i}ve procedure,
of shifting the conduction energies and renormalizes the velocity matrix
elements, which works for the linear response, does
not work at all for the nonlinear response. The new expression we derive for 
$\chi^{abc}(-2\omega;\omega,\omega)$ contains two terms directly obtained from the scissors
Hamiltonian that are clearly required to obtain gauge invariance within the
scissors implementation. Earlier, Nastos et al.\cite{nastosPRB05} showed the
correct way of calculating $\chi^{abc}(-2\omega;\omega,\omega)$ using the scissors Hamiltonian
within the length-gauge, and in the present article we can identify a unified
approach to the calculation of $\chi^{abc}(-2\omega;\omega,\omega)$, 
with and without the scissors
correction, that gives the same result for the velocity-gauge~and length-gauge. While we
can verify this analytically for linear response,\cite{aversaPRB95}
 for the SHG response
coefficient we can only confirm it numerically. Nonetheless, with this
confidence acquired our approach can serve as a model for the
gauge-invariant calculation of other nonlinear optical response coefficients.

The article is organized as follows. In Sec. \ref{theory} we present the
most important steps involved in the derivation of second order
susceptibility tensor $\chi^{abc}(-2\omega;\omega,\omega)$, within the length-gauge~and velocity-gauge~approaches.
In Sec. \ref{results} we show the results of the numerical evaluation
of $\chi^{abc}(-2\omega;\omega,\omega)$, taking as an example the zinc-blende bulk semiconductor
GaAs, and discuss them. 
We calculate the expressions for $\chi^{abc}(-2\omega;\omega,\omega)$ with 
{\it ab initio} programs 
based on density function theory within the local density
approximation 
(DFT-LDA), using all-electron and pseudopotential schemes.
Finally, in Sec. \ref{conclusions} we present our conclusions.

\section{Theory}

\label{theory}

In this section we present the strategy used to calculate the second-order
nonlinear response.
Although this has already been discussed in earlier studies, we consider
both the velocity gauge and the length gauge response
within a common formalism. Our derivation includes new terms not included
before in the velocity gauge, for both the linear and the nonlinear
response. For the nonlinear response, the new terms are crucial for
establishing 
numerically that both gauges give the same result, as they must. 

\subsection{Perturbation approach}

We use the independent particle approximation and neglect local field
and excitonic effects and
treat the electromagnetic field classically, while the matter is
described quantum-mechanically.
We can describe the system using a scaled one electron density
operator ${\rho}$, with which we can calculate the expectation value of a
single-particle observable $\mathcal{O}$ as 
$\langle{\cal O}\rangle=\mbox{Tr}({\rho} {\cal O}),  
\label{calo}$
with $\mathcal{O}$ the associated quantum mechanical operator and Tr
the trace. 
The density operator satisfies
$
i\hbar (d{\rho}/dt)=[H(t),{\rho}],  
\label{rho}
$
with $H(t)$ as the total single electron Hamiltonian, written as 
\begin{equation*}
H(t)=H_{0}+H_{I}(t),  
\label{ache}
\end{equation*}
where $H_{0}$ is the unperturbed time-independent Hamiltonian, and $H_{I}(t)$
is the time-dependent potential energy due to the interaction of the
electron with the electromagnetic field; $H_{0}$ has eigenvalues
$\hbar \omega_{n}(\mathbf{k})$
and eigenstates $| n\mathbf{k} \rangle$ (Bloch states) labeled by a band 
index $n$ and crystal momentum $\mathbf{k}$.
To proceed with the solution of $\rho$  it is convenient to use the
interaction picture, where a unitary operator $U=\exp ({iH_{0}t/\hbar })$
transforms any operator $\mathcal{O}$ into $\tilde{\mathcal{O}}=U\mathcal{\ O
}U^{\dagger }$. Even if $\mathcal{O}$ does not depend on $t$, $\tilde{
\mathcal{O}}$ does through the explicit time dependence of $U$. 
The dynamical 
equation for $\tilde{\rho}$ is 
given by
\begin{equation*}
i\hbar \frac{d\tilde{{\rho}}}{dt}=[\tilde{H}_{I}(t),\tilde{{\rho}}],  
\label{rho1}
\end{equation*}
with solution 
\begin{equation}
i\hbar \tilde{{\rho}}(t)=i\hbar \tilde{{\rho}}_{0}+\int_{-\infty }^{t}dt^{\prime }[
\tilde{H}_{I}(t^{\prime }),\tilde{\rho}(t^{\prime })],  
\label{trans}
\end{equation}
where $\tilde{\rho}_{0}=\tilde{\rho}(t=-\infty )$ is the unperturbed density matrix. We look
for the standard perturbation series solution, 
$\tilde{\rho}(t)=\tilde{\rho}^{(0)}+\tilde{\rho}^{(1)}+\tilde{\rho}^{(2)}+\cdots$,  
where the superscript denotes the order (power) with which each term depends
on the perturbation $H_{I}(t)$. From Eq.~\eqref{trans} the $N$-th order
term is 
\begin{equation}
\tilde{{\rho}}^{(N)}(t)=\frac{1}{i\hbar }\int_{-\infty }^{t}dt^{\prime }[\tilde{
H}_{I}(t^{\prime }),\tilde{\rho}^{(N-1)}(t^{\prime })].  
\label{rhop}
\end{equation}
The series is generated by the unperturbed density operator $\tilde{\rho}
^{(0)}\equiv \tilde{\rho}_{0}$, assumed to be the diagonal Fermi-Dirac distribution, 
$\langle n\mathbf{k}|\tilde{\rho}_{0}|n\mathbf{k}\rangle=f(\hbar \omega_{n}(\mathbf{k}))\equiv f_{n}$. For a
clean, cold semiconductor $f_{n}=1$ for $n$ a valence ($v$) or
occupied band and zero for $n$ a conduction ($c$) or empty band. \ This we
assume throughout.
We remark that the expectation values satisfy
$
\langle{\cal O}\rangle=
\mbox{Tr}({\rho}{\cal O})
=
\mbox{Tr}(\tilde{{\rho}}\tilde{\cal O})
$.  

We first look for the expectation value of the macroscopic current density, 
$\mathbf{J}$, given by 
\begin{equation}
\langle{\mathbf{J}}\rangle=\frac{e}{\Omega}\mbox{Tr}({\rho}\dot{\mathbf{r}})
,
\label{pe}
\end{equation}
where $\dot{\mathbf{r}}$ is the time derivative of the position operator of the
electron of charge $e$, 
\begin{equation}
\mathbf{v}\equiv \dot{\mathbf{r}}=\frac{1}{i\hbar }[\mathbf{r},H],  
\label{mv}
\end{equation}
with $\mathbf{v}$ the velocity operator of the electron, and $\Omega$ the
normalization volume. We calculate the macroscopic polarization density 
$\mathbf{P}$, related to $\langle{\mathbf{J}}\rangle$ by $\langle{\mathbf{J}}\rangle=d\mathbf{P}/dt$. For a
perturbing (Maxwell macroscopic) electromagnetic field, $\mathbf{E}(t)=
\mathbf{E}(\omega )e^{-i\tilde{\omega} t}+c.c.$,
where $\tilde\omega=\omega+i\eta $,
and $\eta >0$ is used to adiabatically turn on the interaction,
we write the second order nonlinear
polarization as, 
\begin{equation}
P^{a(2)}(2\omega)=\chi ^{abc}(-2\omega;\omega,\omega)E^{b}(\omega)E^{c}(\omega),  
\label{pshg}
\end{equation}
where $\chi^{abc}(-2\omega ;\omega ,\omega )$ is the nonlinear
susceptibility responsible of second harmonic generation (SHG). The 
superscripts in Eq.~\eqref{pshg} denote Cartesian components, and if
repeated are to be summed over. Without loss of generality we can always
define $\chi^{abc}(-2\omega;\omega,\omega)$ to satisfy intrinsic permutation
symmetry, 
$\chi^{abc}(-2\omega ;\omega ,\omega )=\chi ^{acb}(-2\omega ;\omega ,\omega )$;
if it did not, the part that did not satisfy intrinsic permutation symmetric
would make no contribution to the second order polarization. This part
could be dropped since it would
have no physical significance.

The unperturbed Hamiltonian is given by 
\begin{equation}\label{h0}
H_{0}=\frac{p^{2}}{2m_e}+V(\mathbf{r}),
\end{equation}
with $m_e$ the mass of the electron, $\mathbf{p}$ its canonical momentum, and 
$V(\mathbf{r})$ the local periodic crystal potential, where we neglect spin-orbit terms.
This Hamiltonian is used to solve the Kohn-Sham equations\cite{kohnPR65} of Density 
Functional Theory (DFT), for convenience usually within the Local
Density Approximation (LDA).
As is well known, the use of these solutions as single 
particle states leads to an underestimation of the band gap. A standard
procedure to correct for this is to
use the so-called ``scissors approximation'', by which one rigidly shifts the
conduction bands in energy so that the band gap corresponds to the accepted
experimental band gap; this is often in fairly good agreement with the GW
band gap based on a more sophisticated calculation.\cite{hybertsenPRB86}
Concurrently, one uses the LDA wave functions, since they produce band
structures with dispersion relations similar to those predicted by the GW
approximation. Mathematically, one adds the scissors (non-local) term 
$S(\mathbf{r},\mathbf{p})$, to the unperturbed or unscissored Hamiltonian $H_{0}$, i.e. 
\begin{equation*}
H_{0}^{S}=H_{0}+S(\mathbf{r},\mathbf{p}),
\end{equation*}
where 
\begin{equation}
S(\mathbf{r},\mathbf{p})=\hbar \Delta\sum_{n}\int d^{3}k(1-f_{n})
|n\mathbf{k}\rangle\langle n\mathbf{k}|,
\label{hats}
\end{equation}
with $\hbar \Delta$  the rigid ($\mathbf{k}$-independent) energy correction to be
applied. 
Several properties of $S(\mathbf{r},\mathbf{p})$ are shown in the
appendix.
The unscissored and scissored Hamiltonians satisfy 
\begin{eqnarray*}
H_{0}\psi _{n\mathbf{k}}(\mathbf{r}) &=&\hbar \omega_{n}(\mathbf{k})\psi _{n\mathbf{k}}(\mathbf{r}),
\label{hamils} \\
H_{0}^{S}\psi _{n\mathbf{k}}(\mathbf{r}) &=&\hbar \omega_{n}^{S}(\mathbf{k})\psi _{n\mathbf{k}}(\mathbf{r}),
\end{eqnarray*}
where 
\begin{equation}
\omega_{n}^{S}(\mathbf{k})=\omega_{n}(\mathbf{k})+(1-f_{n})\Delta,  
\label{wese}
\end{equation}
and $\psi _{n\mathbf{k}}(\mathbf{r})=\langle\mathbf{r} | n\mathbf{k}\rangle$ is the coordinate
representation of the ket $|n\mathbf{k}\rangle$. We emphasize that the scissored
Hamiltonian has the same eigenfunctions as the unscissored Hamiltonian.

\subsection{Velocity gauge Formalism}

To calculate the optical response in the velocity gauge, we use the minimal
substitution through which, in the presence of an electromagnetic field, the
Hamiltonian is written as 
\begin{equation}
H^{S}=\frac{1}{2m_e}(\mathbf{p}-\frac{e}{c}\mathbf{A})^{2}+V(\mathbf{r})+S(\mathbf{r},\mathbf{p}-\frac{e}{c}
\mathbf{A}),  
\label{hsa}
\end{equation}
where $\mathbf{A}$ is the vector potential; one obtains the magnetic field as $\mathbf{B}
=\nabla \times \mathbf{A}$ and the electric field as $\mathbf{E}=-(1/c)\dot{\mathbf{A}}$, with 
$c$ the speed of light in vacuum. In general these electric and magnetic
fields are taken to be the macroscopic Maxwell fields. We assume the
long-wavelength limit, in which $\mathbf{A}$ is uniform and only depends on time.
Furthermore, we take a harmonic perturbation of the form 
$\mathbf{A}(t)=\mathbf{A}(\omega)e^{-i\tilde\omega t}+\mathbf{A}^{\ast }(\omega)e^{i\tilde\omega^{\ast }t}$, 
where
only the
``positive frequency'' term will be kept in
the following, because that term will contribute to the positive frequency
part of the linear and second harmonic responses. Expanding the scissors
operator according to,\cite{nastosPRB05} 
\begin{equation*}
S(\mathbf{r},\mathbf{p}-\frac{e}{c}\mathbf{A})=S(\mathbf{r},\mathbf{p})+\frac{e}{c}\frac{i}{\hbar }\mathbf{A}
\cdot \lbrack \mathbf{r},S(\mathbf{r},\mathbf{p})]+\frac{1}{2!}\left( \frac{e}{c}\frac{i}{
\hbar }\right) ^{2}[\mathbf{A}\cdot \mathbf{r},[\mathbf{A}\cdot \mathbf{r},S(\mathbf{r},\mathbf{p})]]+\cdots ,
\label{sc}
\end{equation*}
leads to the following scissored Hamiltonian up to second order in $\mathbf{A}$
\begin{equation*}
H^{S}=H_{0}^{S}+H_{I,1}+H_{I,2},
\end{equation*}
where 
\begin{eqnarray}
H_{I,1} &=&-\frac{e}{c}\mathbf{A}\cdot \mathbf{v}^{\Sigma},  \label{hsnll} \\
H_{I,2} &=&-\frac{ie^{2}}{2\hbar c^{2}}[r^{b},v^{S,c}]A^{b}A^{c}+\frac{e^{2}
}{2m_ec^{2}}A^{2},
\end{eqnarray}
are the linear and nonlinear (second order) interaction Hamiltonians. The 
$e^{2}A^{2}/(2m_ec^{2})$ term is only a function of time contributing
to a global phase factor to the electron wave function that has
no effect on expectation values, so it can be dropped. We
have defined 
\begin{equation}
\mathbf{v}^{S}=-\frac{i}{\hbar }[\mathbf{r},S(\mathbf{r},\mathbf{p})],  
\label{vnl}
\end{equation}
as the contribution to the velocity operator due to the non-local scissors
term, and 
\begin{equation}
\mathbf{v}^{\Sigma }=\frac{\mathbf{p}}{m_e}+\mathbf{v}^{S},  
\label{vs}
\end{equation}
as the scissored velocity operator. 
 From Eq.~\eqref{mv} the current operator 
$\mathbf{j}=e\dot{\mathbf{r}}$, up to second order in $\mathbf{A}$, is 
\begin{equation*}
j^{a}=j_{0}^{a}+j_{1}^{a}+j_{2}^{a},  \label{j2nl}
\end{equation*}
with 
\begin{eqnarray*}
j_{0}^{a} &=&ev^{\Sigma ,a},  \label{jotas} \\
j_{1}^{a} &=&-\frac{e^{2}}{cm_e}A^{a}+\frac{ie^{2}}{\hbar c}
[r^{a},v^{S,b}]A^{b}, \\
j_{2}^{a} &=&-\frac{e^{3}}{2\hbar ^{2}c^{2}}
[r^{a},[r^{b},v^{S,c}]]A^{b}A^{c},
\end{eqnarray*}
operators of zero, first and second order in $\mathbf{A}$, respectively. 
From Eq.~\eqref{pe}, 
\begin{equation}
\langle{J}^{(1)a}\rangle=\frac{1}{\Omega}\mbox{Tr}(j_{0}^{a}{\rho}^{(1)})+\frac{1}{\Omega}
\mbox{Tr}(j_{1}^{a}{\rho}^{(0)}),  \label{jmacl}
\end{equation}
is the linear macroscopic current density, and 
\begin{equation}
\langle{J}^{(2)a}\rangle=\frac{1}{\Omega}\mbox{Tr}(j_{0}^{a}{\rho}^{(2)})+\frac{1}{\Omega}
\mbox{Tr}(j_{1}^{a}{\rho}^{(1)})+\frac{1}{\Omega}\mbox{Tr}(j_{2}^{a}{\rho}^{(0)}),
\label{jmacnl}
\end{equation}
is the nonlinear (second order) macroscopic current density.

\subsubsection{Linear Response}

We calculate the linear response, within the velocity gauge, and show that
there is a new term not previously included when the scissored Hamiltonian
is used. 
Indeed, we show that by coincidence the 
``usual'' way of including the scissor correction leads to
the correct result. 
That is, the scissors correction only gives a rigid
shift in the energy axis of the unscissored spectrum by an amount equal to 
$\Delta$; the line shape of the spectrum is the same for both the scissored and
the unscissored
Hamiltonians.\cite{nastosPRB05,solePRB93,levinePRB91,levinePRL89} 
In the following, we show
that if the usual procedure is used
for the nonlinear response the resulting 
scissored susceptibility is wrong.
The derivation of
the linear response here is important for making sense of our later results,
and also sets some of the intermediate results that will be used in the
calculation of the nonlinear response. 
In passing, we also show that the linear response is gauge invariant, since we 
obtain the same analytic result for the linear susceptibility in both gauges.
This agreement holds with and without the scissors correction.

We start by taking matrix elements of Eq.~\eqref{rhop} to obtain 
\begin{equation*}
\tilde{{\rho}}_{mn}^{(1)}(t)=\frac{ei}{\hbar c}\int_{-\infty
}^{t}dt^{\prime} A^b(t')
\sum_{\ell }\left( {\tilde{v}}_{m\ell }^{\Sigma ,b}(t^{\prime
}){\tilde{{\rho}}}_{\ell n}^{(0)}(t^{\prime })-{\tilde{{\rho}}}_{m\ell
}^{(0)}(t^{\prime }){\tilde{v}}_{\ell n}^{\Sigma ,b}(t^{\prime })\right) ,
\end{equation*}
where the sum over $\ell $ is over all states, and we have used 
Eq.~\eqref{hsnll}. Since $U(t)=\exp (iH_{0}^{S}t/\hbar )$, we get 
$
{\tilde v^{\Sigma,b}}_{m\ell}(t')
=
v^{\Sigma,b}_{m\ell}
e^{i\omega^S_{m\ell}t'}
$,
${\tilde{{\rho}}^{0}}_{\ell n}(t^{\prime })=f_{\ell }\delta_{\ell n}$, and
$\omega_{m\ell }^{S}=\omega_{m}^{S}(\mathbf{k})-\omega_{\ell }^{S}(\mathbf{k})$, where we have
omitted the dependence on $\mathbf{k}$ from the already crowded notation. Then
$\tilde{\rho}_{mn}^{(1)}(t)={\rho}_{mn}^{(1)}
e^{i\omega^S_{mn}t}  
e^{-i\tilde{\omega} t}
$,
with
\begin{eqnarray}\label{rho3}
\rho_{mn}^{(1)}(t)
=
\frac{e}{\hbar c}
\frac{v_{mn}^{\Sigma,b}f_{nm}}{
\omega^S_{mn}-\tilde\omega}
A^b(\omega)
,
\label{rhotn}
\end{eqnarray} 
where $f_{nm}=f_n-f_m$.
Using
$\mbox{Tr}({\rho}^{(0)})/\Omega=n_0$,
with $n_0$ the electronic density,
and $d\mathbf{P}/dt=\left\langle \mathbf{J}
\right\rangle $ to write $P^{a}(\omega)=(i/\tilde\omega)\left\langle J^{a}(\omega
)\right\rangle =\chi ^{ab}(-\omega;\omega)E^{b}(\omega)$, we get 
from Eq.~\eqref{jmacl} that
\begin{eqnarray}
\chi ^{ab}(-\omega;\omega) &=&\frac{e^{2}}{\hbar \tilde\omega^{2}}\int \frac{d^{3}k}{8\pi
^{3}}\sum_{m\ne n}\frac{v_{nm}^{\Sigma ,a}v_{mn}^{\Sigma ,b}f_{nm}}{\omega_{mn}^{S}-
\tilde\omega}-\frac{e^{2}n_0}{m_e\tilde\omega^{2}}\delta_{ab}+\frac{ie^{2}}{\hbar \tilde\omega^{2}}\frac{1}{
\Omega}\mbox{Tr}({\rho}^{(0)}\mathcal{F}^{ab})  \notag  \label{chi1} \\
&=&\frac{e^{2}}{\hbar }\int \frac{d^{3}k}{8\pi ^{3}}\sum_{m\ne n}f_{nm}v_{nm}^{
\Sigma ,a}v_{mn}^{\Sigma ,b}\Big(\frac{1}{(\omega_{mn}^{S})^{2}(\omega_{mn}^{S}-
\tilde\omega)}+\frac{1}{(\omega_{mn}^{S})^{2}\tilde\omega}+\frac{1}{\omega_{mn}^{S}\tilde\omega^{2}}\Big) 
\notag \\
&-&\frac{e^{2}n_0}{m_e\tilde\omega^{2}}\delta_{ab}+\frac{ie^{2}}{\hbar \tilde\omega^{2}}\frac{1}{\Omega
}\mbox{Tr}({\rho}^{(0)}\mathcal{F}^{ab}),
\end{eqnarray}
is the linear susceptibility within the scissored Hamiltonian; we used a
partial fraction expansion in the first term after the first equal sign.
We have defined 
\begin{equation}
\mathcal{F}^{ab}=[r^{a},v^{S,b}]
,
\label{calefe}
\end{equation}
used the fact that the $f_{nm}$ factor
allow us to write $m\ne n$
and that in the continuous limit of $\mathbf{k}$
$(1/\Omega)\sum_\mathbf{k}\to\int d^3k/(8\pi^3)$.

From time reversal symmetry we have that
$v_{mn}^{S}(-\mathbf{k})=-v_{nm}^{S}(\mathbf{k})$
and $\omega_{mn}^{S}(-\mathbf{k})=\omega_{mn}^{S}(\mathbf{k})$ with
which it follows
that the contribution to $\chi^{ab}(-\omega;\omega)$ coming
from the $1/\tilde\omega$ cancels out.
By simple subindex manipulation, the third term, combined with the fourth
term in the right hand side of Eq.~\eqref{chi1}, gives 
\begin{equation}
\label{trd2}
\frac{e^{2}}{\hbar }\int \frac{d^{3}k}{8\pi ^{3}}
\sum_{m\ne n}f_n\frac{v_{nm}^{\Sigma ,a}v_{mn}^{\Sigma ,b}+v_{mn}^{\Sigma
,a}v_{nm}^{\Sigma ,b}}{\omega_{mn}^{S}}-\frac{e^{2}n_0}{m_e}\delta_{ab}
\equiv 
\zeta ^{ab}.
\end{equation}
The last term on the right hand side of Eq.~\eqref{chi1} reduces to 
\begin{eqnarray}\label{eta}
\frac{ie^{2}}{\hbar}\int \frac{d^{3}k}{8\pi ^{3}}\sum_{n}f_{n}
\mathcal{F}_{nn}^{ab}  
\equiv \eta ^{ab}
\end{eqnarray}
where
\begin{equation}
\mathcal{F}_{nn}^{ab}=i\Delta\sum_{m(\neq n)}f_{nm }(r_{nm }^{a}r_{mn}^{b}+r_{nm}^{b}r_{mn}^{a}),  \label{calfnn}
\end{equation}
summing $m$ over all $v$ and $c$ states different from $n$ (see the Appendix). 
Finally, Eq.~\eqref{chi1} reduces to 
\begin{equation}
\label{chi2}
\chi ^{ab}(-\omega;\omega)=\frac{e^{2}}{\hbar }\int \frac{d^{3}k}{8\pi ^{3}}
\sum_{m\ne n}\frac{v_{nm}^{\Sigma ,a}v_{mn}^{\Sigma ,b}f_{nm}}{(\omega
_{mn}^{S})^{2}(\omega_{mn}^{S}-\tilde\omega)}
+
\frac{\zeta ^{ab}}{\tilde\omega^{2}}
+
\frac{\eta^{ab}}{\tilde\omega^{2}}
,
\end{equation}
which is the linear response coefficient obtained within
the velocity gauge, including the
scissors correction. 
Using
\begin{equation}
\mathbf{v}_{nm}^{\Sigma}=\frac{\omega_{nm}^{S}}{\omega_{nm}}\mathbf{v}_{nm}\quad (n\neq m),
\label{vsn}
\end{equation}
and $\omega_{mn}^{S}=\omega_{mn}-f_{mn}\Delta$, from the appendix we
get that
\begin{equation}
\chi ^{ab}(-\omega;\omega)=\frac{e^{2}}{\hbar }\int \frac{d^{3}k}{8\pi ^{3}}
\sum_{m\ne n}\frac{f_{nm}v_{nm}^{a}v_{mn}^{b}}{\omega_{mn}^{2}(\omega_{mn}^{S}-\tilde\omega)}-
\frac{e^{2}}{\tilde\omega^{2}}\int \frac{d^{3}k}{8\pi ^{3}}\sum_{n}f_{n}\left[ \frac{
1}{m_{n}^{\ast }}\right] ^{ab},
\label{lastb}
\end{equation}
where $[1/m^*_n]^{ab}$ is the effective mass tensor given in Eq.~\eqref{meff}.

For a clean, cold semiconductor $f_n=f_n(\mathbf{k})=1$ or 0, independent of
$\mathbf{k}$ and the integration over the Brillouin Zone of
the term involving the effective mass tensor vanishes identically,
\cite{ghahramaniPRB91} which implies that 
\begin{equation}\label{chivda} 
\chi ^{ab}(-\omega;\omega)=\frac{e^{2}}{\hbar }\int \frac{d^{3}k}{8\pi ^{3}}
\sum_{m\ne n}\frac{f_{nm}v_{nm}^{a}v_{mn}^{b}}{\omega_{mn}^{2}(\omega_{mn}^{S}-\omega
-i\eta )}
,
\end{equation}
where the energy denominator leads to
resonances when $\omega_{mn}^{S}=\omega$.

A similar calculation neglecting the scissors term in the Hamiltonian leads
to 
\begin{equation*}
\chi _{\mathrm{unscissored}}^{ab}(-\omega;\omega)=\frac{e^{2}}{\hbar }\int \frac{
d^{3}k}{8\pi ^{3}}\sum_{m\ne n}\frac{f_{nm}v_{nm}^{a}v_{mn}^{b}}{\omega_{mn}^{2}(\omega
_{mn}-\omega-i\eta )},
\end{equation*}
where now the resonances are at $\omega
_{mn}=\omega$. 
A na\"{\i}ve procedure to \textquotedblleft
scissors\textquotedblright\ above result would be to take 
\begin{equation}
\label{nai}
\chi _{\mathrm{na\ddot{\i}ve}}^{ab}(-\omega;\omega)=\frac{e^{2}}{\hbar }\int \frac{d^{3}k}{
8\pi ^{3}}\sum_{m\ne n}\frac{f_{nm}v_{nm}^{\Sigma,a}v_{mn}^{\Sigma,b}}{(\omega
_{mn}^{S})^{2}(\omega_{mn}^{S}-\omega-i\eta )},
\end{equation}
an incorrect strategy, since it misses the second and third important terms
on the right hand side of Eq.~\eqref{chi2}. However, using Eq.~\eqref{vsn}
in Eq.~\eqref{nai} leads by coincidence to the correct result 
of Eq.~\eqref{chivda}.
It appears that this point has not been appreciated in the literature; 
Eq.~\eqref{chi2} shows the correct way to include the scissors Hamiltonian
within the velocity gauge, which is not simply the usual strategy
illustrated by Eq.~\eqref{nai}.\cite{last}

Using Eq.~\eqref{remn}, we rewrite Eq.~\eqref{chivda} as 
\begin{equation*}
\chi ^{ab}(-\omega;\omega)=\frac{e^{2}}{\hbar }\int \frac{d^{3}k}{8\pi ^{3}}
\sum_{m\ne n}\frac{f_{nm}r_{nm}^{a}r_{mn}^{b}}{\omega_{mn}^{S}-\omega-i\eta },
\end{equation*}
which is identical to the length gauge result for the scissored 
Hamiltonian.\cite{nastosPRB05} Again, for the unscissored Hamiltonian one 
gets,\cite{sipePRB00} 
\begin{equation*}
\chi _{\mathrm{unscissored}}^{ab}(-\omega;\omega)=\frac{e^{2}}{\hbar }\int \frac{
d^{3}k}{8\pi ^{3}}\sum_{m\ne n}\frac{f_{nm}r_{nm}^{a}r_{mn}^{b}}{\omega_{mn}-\omega
-i\eta },
\end{equation*}
and as discussed by Nastos et al.\cite{nastosPRB05}, in the
length gauge the unscissored linear susceptibility can be \textquotedblleft
scissored\textquotedblright\ by simply shifting $\omega_{nm}\rightarrow \omega
_{nm}^{S}$ and keeping the same matrix elements $\mathbf{r}_{nm}$. Thus, as in the
velocity gauge, the scissored linear response is simply rigidly shifted in
energy form its LDA result, keeping the same line shape. We remark that this
constitutes a direct analytical proof of gauge invariance for the linear
response. For the nonlinear response, we have not been able to construct any
such analytical proof. \ However, we can at least provide a check on the
gauge invariance through a numerical calculation. \ To that end we need
expressions for the second order response in the velocity and length gauges,
and we now turn to the first of these.

\subsubsection{Non Linear Response}

Using the results of Sec. \ref{theory} and the previous subsection, we find
that to second order in $\mathbf{A}$ the density matrix is given by 
\begin{eqnarray*}
\tilde{{\rho}}_{mn}^{(2)}(t) &=&-\frac{i}{\hbar }\int_{-\infty }^{t}dt^{\prime
}[\tilde{H}_{I,1}(t^{\prime }),\tilde{{\rho}}^{(1)}(t^{\prime })]_{mn}-\frac{i}{
\hbar }\int_{-\infty }^{t}dt^{\prime }[\tilde{H}_{I,2}(t^{\prime }),\tilde{
{\rho}}^{(0)}(t^{\prime })]_{mn}  \notag  \label{rho2nl} \\
&=&\frac{e^{2}}{\hbar ^{2}c^{2}}
\Big
[
\sum_{\ell(\neq n)}
\frac{f_{n\ell}v_{m\ell}^{\Sigma
,b}v_{\ell n}^{\Sigma ,c}}{\omega_{\ell n}^{S}-\tilde\omega}
-
\sum_{\ell(\neq m)}
\frac{f_{\ell m}v_{m\ell}^{\Sigma
,c}v_{\ell n}^{\Sigma ,b}}{\omega_{m\ell}^{S}-\tilde\omega}
+\frac{i}{2}f_{nm}\mathcal{F}
_{mn}^{bc}
\Big]\frac{A^{b}(\omega)A^{c}(\omega)}{\omega_{mn}^{S}-2\tilde\omega}
e^{-i2\tilde{\omega}t} 
e^{i\omega_{mn}^{S}t}
\notag \\
&=&{\rho}_{mn}^{(2)}
e^{i\omega_{mn}^{S}t}
, 
\end{eqnarray*}
where, as in the linear response, only the positive frequency terms
are used.
The ${\cal F}^{ab}_{mn}$ term is obtained in the appendix in Eq.~\eqref{calfnmab}.
The macroscopic current density can be calculated through
Eq.~\eqref{jmacnl},
 where we take each term separately: 
\begin{eqnarray}
\frac{1}{\Omega}\mbox{Tr}(j_{0}^{a}{\rho}^{(2)}) &=&e\int \frac{d^{3}k}{8\pi ^{3}}
\sum_{mn}v_{nm}^{\Sigma ,a}{\rho}_{nm}^{(2)}  \notag  \label{tr1} \\
&=&\frac{e^{3}}{\hbar ^{2}c^{2}}\Big[\int \frac{d^{3}k}{8\pi ^{3}}\sum_{mn}
\frac{v_{nm}^{\Sigma ,a}}{\omega_{mn}^{S}-2\tilde\omega}
\Big(
\sum_{\ell(\neq n)}
\frac{f_{n\ell}v_{m\ell}^{\Sigma
,b}v_{\ell n}^{\Sigma ,c}}{\omega_{\ell n}^{S}-\tilde\omega}
-
\sum_{\ell(\neq m)}
\frac{f_{\ell m}v_{m\ell}^{\Sigma
,c}v_{\ell n}^{\Sigma ,b}}{\omega_{m\ell}^{S}-\tilde\omega}
\Big)
\nonumber\\
&+&\frac{i}{2}\int \frac{d^{3}k}{8\pi ^{3}}\sum_{m\ne n}\frac{
f_{nm}v_{nm}^{\Sigma ,a}\mathcal{F}_{mn}^{bc}}{\omega_{mn}^{S}-2\tilde\omega}\Big]
A^{b}(\omega)A^{c}(\omega) 
e^{-i2\tilde{\omega} t}
,
\end{eqnarray}
\begin{eqnarray}
\frac{1}{\Omega}\mbox{Tr}(j_{1}^{a}{\rho}^{(1)}) &=&\frac{e^{2}}{cm_e}\frac{1}{\Omega}
\mbox{Tr}({\rho}^{(1)}) A^{a}(\omega)e^{-i\tilde{\omega} t}
+\frac{ie^{2}}{\hbar c}\frac{1}{\Omega}\mbox{Tr}(
\mathcal{F}^{ab}{\rho}^{(1)})
A^{b}(\omega) e^{-i\tilde{\omega} t}
\notag  \label{tr2} \\
&=&\frac{ie^{3}}{\hbar ^{2}c^{2}}
\int \frac{d^{3}k}{8\pi^3}
\sum_{m\ne n}f_{nm}
\frac{\mathcal{F}_{nm}^{ab}v_{mn}^{\Sigma ,c}}{\omega_{mn}^{S}-\tilde\omega}
A^{b}(\omega)A^{c}(\omega)
e^{-i2\tilde{\omega} t}
,
\end{eqnarray}
since $\mbox{Tr}({\rho}^{(1)})=0$ (see ~\eqref{rhotn}), and finally 
\begin{eqnarray}
\frac{1}{\Omega}\mbox{Tr}(j_{2}^{a}{\rho}^{(0)}) &=&-\frac{e^{3}}{2\hbar ^{2}c^{2}}
\int \frac{d^{3}k}{8\pi^3}
\sum_{mn}{\rho}
_{mn}^{(0)}[r^{a},[r^{b},v^{S,c}]]_{nm}
A^{b}(\omega)A^{c}(\omega)
e^{-i2\tilde{\omega} t}
\notag  \label{tr33bb} \\
&=&-\frac{e^{3}}{2\hbar ^{2}c^{2}}\int \frac{d^{3}k}{8\pi ^{3}}
\sum_{n}f_{n}[r^{a},\mathcal{F}^{bc}]_{nn}
A^{b}(\omega)A^{c}(\omega)
e^{-i2\tilde{\omega} t}
\notag \\
&=&-\frac{e^{3}}{2\hbar ^{2}c^{2}}\int \frac{d^{3}k}{8\pi ^{3}}\sum_{n}f_{n}
\Big(r_{nm}^{a}\mathcal{F}_{mn}^{bc}-\mathcal{F}_{nm}^{bc}r_{mn}^{a}+i\frac{
\partial }{\partial k^{a}}\mathcal{F}_{e,nn}^{bc}\Big)
A^{b}(\omega)A^{c}(\omega)
e^{-i2\tilde{\omega} t}
,
\end{eqnarray}
where we used the expression for $[r^{a},\mathcal{F}^{bc}]_{nn}$ derived in
the Appendix. Again employing time reversal symmetry, we can take $
r_{nm}^{a}(-\mathbf{k})=r_{mn}^{a}(\mathbf{k})$, $\mathbf{r}_{mn;\mathbf{k}}(\mathbf{k})=-\mathbf{r}_{nm;\mathbf{k}}(-
\mathbf{k})$,
$\mathcal{F}_{nm}^{bc}(-\mathbf{k})=\mathcal{F}_{mn}^{bc}(\mathbf{k})$,
and $\mathcal{F}_{nm}^{bc^{\ast }}(\mathbf{k})=-\mathcal{F}
_{mn}^{bc}(\mathbf{k})$. If we add the $\mathbf{k}$ and the $-\mathbf{k}$ contribution
in Eq.~\eqref{tr33bb}, we get a perfect cancellation of the terms within the
parenthesis, and so the contribution from $\mbox{Tr}(j_{2}^{a}{\rho}^{(0)})$
vanishes.

Using $\left\langle \mathbf{J}\right\rangle ^{(2)}=d\mathbf{P}^{(2)}/dt$ for the second
harmonic response, we get $\mathbf{P}^{(2)}(2\omega)=(i/2\tilde\omega)\left\langle \mathbf{J}
^{(2)}(2\omega)\right\rangle $, and from Eq.~\eqref{pshg}, Eq.~\eqref{tr1} and
Eq.~\eqref{tr2} we find 
\begin{eqnarray*}
\chi ^{abc}(-2\omega;\omega,\omega) &=&\frac{e^{3}}{2\hbar ^{2}\tilde\omega^{3}}\Big[-i\int 
\frac{d^{3}k}{8\pi ^{3}}\sum_{mn}\frac{v_{nm}^{\Sigma ,a}\{v_{m\ell
}^{\Sigma ,b}v_{\ell n}^{\Sigma ,c}\}}{\omega_{mn}^{S}-2\tilde\omega}
\Big(
\sum_{\ell(\neq n)}
\frac{f_{n\ell }}{\omega_{\ell n}^{S}-\tilde\omega}
-
\sum_{\ell(\neq m)}
\frac{f_{\ell m}}{\omega_{m\ell }^{S}-\tilde\omega}
\Big)  
\notag  \label{chi2tot} \\
&+&\frac{1}{2}\int \frac{d^{3}k}{8\pi ^{3}}\sum_{m\neq n}f_{nm}\Big(\frac{
v_{nm}^{\Sigma ,a}\{\mathcal{F}_{mn}^{bc}\}}{\omega_{mn}^{S}-2\tilde\omega}+2\frac{\{
\mathcal{F}_{nm}^{ab}v_{mn}^{\Sigma ,c}\}}{\omega_{mn}^{S}-\tilde\omega}\Big)\Big],
\end{eqnarray*}
where $\{\}$ implies the symmetrization of the Cartesian indices $bc$, i.e. $
\{u^{b}s^{c}\}=(u^{b}s^{c}+u^{c}s^{b})/2$. We take half of this expression
and add to it the corresponding expression with $\mathbf{k}$ replaced by $-
\mathbf{k}$ in the integrand; this of course give a result equivalent to our first
expression, and using time-reversal symmetry we simplify it to yield 
\begin{eqnarray}
\chi ^{abc}(-2\omega;\omega,\omega) &=&\frac{e^{3}}{2\hbar ^{2}\tilde\omega^{3}}\Big[\int 
\frac{d^{3}k}{8\pi ^{3}}\sum_{(mn)\neq\ell }\frac{\mbox{Im}\lbrack v_{nm}^{\Sigma
,a}\{v_{m\ell }^{\Sigma ,b}v_{\ell n}^{\Sigma ,c}\}]}{\omega_{mn}^{S}-2\tilde\omega}
\Big(\frac{f_{n\ell }}{\omega_{\ell n}^{S}-\tilde\omega}-\frac{f_{\ell m}}{\omega_{m\ell
}^{S}-\tilde\omega}\Big)  \notag  \label{chi2tot2} \\
&+&\frac{1}{2}\int \frac{d^{3}k}{8\pi ^{3}}\sum_{m\neq n}f_{nm}\Big(\frac{
\mbox{Re}\lbrack \ v_{nm}^{\Sigma ,a}\{\mathcal{F}_{mn}^{bc}\}]}{\omega
_{mn}^{S}-2\tilde\omega}+2\frac{\mbox{Re}\lbrack \{\mathcal{F}_{nm}^{ab}v_{mn}^{
\Sigma ,c}\}]}{\omega_{mn}^{S}-\tilde\omega}\Big)\Big].
\end{eqnarray}

Following Ghahramani et al.,\cite{ghahramaniPRB91} we use
partial fractions to write the energy denominator of the first term on the
right hand side of Eq.~\eqref{chi2tot2} as 
\begin{equation}
\frac{A}{\tilde\omega^{3}}+\frac{B}{\tilde\omega^{2}}+\frac{C}{\tilde\omega}+F,  \label{parti}
\end{equation}
where the odd terms in $\omega$, $A$ and $C$, can be shown to give zero
contribution.\cite{ghahramaniPRB91} For the second term on the right hand
side of Eq.~\eqref{chi2tot2} we expand the denominators in partial fractions
to obtain 
\begin{equation}
\frac{1}{\tilde\omega^{3}(\omega_{mn}^{S}-\tilde\omega)}=\frac{1}{\tilde\omega(\omega_{mn}^{S})^{3}}+\frac{
1}{\tilde\omega^{3}\omega_{mn}^{S}}+\frac{1}{\tilde\omega^{2}(\omega_{mn}^{S})^{2}}+\frac{1}{(\omega
_{mn}^{S})^{3}(\omega_{mn}^{S}-\tilde\omega)},  \label{frac1}
\end{equation}
and 
\begin{equation}
\frac{1}{\tilde\omega^{3}(\omega_{mn}^{S}-2\tilde\omega)}=\frac{4}{\tilde\omega(\omega_{mn}^{S})^{3}}+
\frac{1}{\tilde\omega^{3}\omega_{mn}^{S}}+\frac{2}{\tilde\omega^{2}(\omega_{mn}^{S})^{2}}+\frac{8}{
(\omega_{mn}^{S})^{3}(\omega_{mn}^{S}-2\tilde\omega)}.  \label{frac2}
\end{equation}
Using time reversal symmetry and simple manipulation of the band indices, we
can show that all the odd terms in $\tilde\omega$ coming from Eq.~\eqref{frac1} and
Eq.~\eqref{frac2} give zero contribution. Collecting the $B$ and $F$ terms of
Eq.~\eqref{parti}, and the non-zero terms of Eq.~\eqref{frac1} and 
Eq.~\eqref{frac2} we obtain 
\begin{eqnarray}
\chi ^{abc}(-2\omega;\omega,\omega) &=&\frac{e^{3}}{2\hbar ^{2}}\int \frac{d^{3}k}{
8\pi ^{3}}\Big[\sum_{n(m\neq\ell)}\Big(\frac{\mbox{Im}\lbrack v_{mn}^{\Sigma
,a}\{v_{n\ell }^{\Sigma ,b}v_{\ell m}^{\Sigma ,c}\}]}{\omega_{nm}^{S}-2\omega
_{\ell m}^{S}}-\frac{\mbox{Im}\lbrack v_{n\ell }^{\Sigma ,a}\{v_{\ell
m}^{\Sigma ,b}v_{mn}^{\Sigma ,c}\}]}{\omega_{\ell n}^{S}-2\omega_{\ell m}^{S}}\Big)
\frac{f_{m\ell }}{(\omega_{\ell m}^{S})^{3}(\omega_{\ell m}^{S}-\tilde\omega)}  \notag
\label{chinn} \\
&-&16
\sum_{\ell(m\neq n)}
\frac{f_{mn}}{\omega_{\ell m}^{S}-2\omega_{nm}^{S}}
\frac{\mbox{Im}\lbrack v_{m\ell }^{\Sigma ,a}\{v_{\ell n}^{\Sigma,b}v_{nm}^{\Sigma ,c}\}]}
{(\omega_{\ell m}^{S})^{3}(\omega_{\ell m}^{S}-2\tilde\omega)}
-16
\sum_{m(\ell\neq n)}
\frac{f_{\ell n}}{\omega_{\ell m}^{S}-2\omega_{\ell n}^{S}}
\frac{\mbox{Im}\lbrack v_{m\ell }^{\Sigma ,a}\{v_{\ell n}^{\Sigma,b}v_{nm}^{\Sigma ,c}\}]}
{(\omega_{\ell m}^{S})^{3}(\omega_{\ell m}^{S}-2\tilde\omega)}  \notag \\
&+&\sum_{m\neq n}\frac{f_{nm}}{(\omega_{mn}^{S})^{3}}\Big(4\frac{\mbox{Re}\lbrack \
v_{nm}^{\Sigma ,a}\{\mathcal{F}_{mn}^{bc}\}]}{\omega_{mn}^{S}-2\tilde\omega}+\frac{
\mbox{Re}\lbrack \{\mathcal{F}_{nm}^{ab}v_{mn}^{\Sigma ,c}\}]}{\omega_{mn}^{S}-
\tilde\omega}\Big)\Big],
\end{eqnarray}
as the nondivergent contribution, to which the divergent term 
\begin{eqnarray*}
\chi _{D}^{abc}(-2\omega;\omega,\omega) &=&\frac{e^{3}}{2\hbar ^{2}\tilde\omega^{2}}\int 
\frac{d^{3}k}{8\pi ^{3}}\Big[\sum_{(mn)\neq\ell}b_{\ell mn}\mbox{Im}\lbrack
v_{nm}^{\Sigma ,a}\{v_{m\ell }^{\Sigma ,b}v_{\ell n}^{\Sigma ,c}\}]  \notag
\label{chi2d} \\
&+&\sum_{m\neq n}\frac{f_{nm}}{(\omega_{mn}^{S})^{2}}\Big(\mbox{Re}\lbrack \
v_{nm}^{\Sigma ,a}\{\mathcal{F}_{mn}^{bc}\}]+\mbox{Re}\lbrack \{\mathcal{F}
_{nm}^{ab}v_{mn}^{\Sigma ,c}\}]\Big)\Big],
\end{eqnarray*}
must be added, where 
\begin{equation*}
b_{\ell mn}=\frac{f_{m\ell }}{\omega_{nm}^{S}\omega_{\ell m}^{S}}\left( \frac{2}{
\omega_{nm}^{S}}+\frac{1}{\omega_{\ell m}^{S}}\right) +\frac{f_{n\ell }}{\omega
_{nm}^{S}\omega_{n\ell }^{S}}\left( \frac{2}{\omega_{nm}^{S}}+\frac{1}{\omega_{n\ell
}^{S}}\right) ,
\end{equation*}
comes from the $B$ term of Eq.~\eqref{parti}. Following the steps of
Ghahramani et al.,\cite{ghahramaniPRB91} we can show that for
a clean, cold semiconductor $\chi _{D}^{abc}(-2\omega;\omega,\omega)=0$.\cite{cabellostesis}

Finally, we insert the explicit values for the $f_n$ factors and take the limit of $\eta
\rightarrow 0$ in Eq .~\eqref{chinn} to find 
\begin{eqnarray}
\mbox{Im}\lbrack \chi _{\mathrm{v}}^{abc}(-2\omega;\omega,\omega)] &=&\frac{\pi
|e|^{3}}{2\hbar ^{2}}\int \frac{d^{3}k}{8\pi ^{3}}\Big[\sum_{vc}\frac{16}{(
\omega_{cv}^{S})^{3}}\Big(\sum_{c^{\prime }}\frac{\mbox{Im}\lbrack
v_{vc}^{\Sigma ,a}\{v_{cc^{\prime }}^{\Sigma ,b}v_{c^{\prime }v}^{\Sigma
,c}\}]}{\omega_{cv}^{S}-2\omega_{c^{\prime }v}^{S}}  \notag  \label{imchicf} \\
&-&\sum_{v^{\prime }}\frac{\mbox{Im}\lbrack v_{vc}^{\Sigma
,a}\{v_{cv^{\prime }}^{\Sigma ,b}v_{v^{\prime }v}^{\Sigma ,c}\}]}{\omega
_{cv}^{S}-2\omega_{cv^{\prime }}^{S}}\Big)\delta(\omega_{cv}^{S}-2\omega)  \notag \\
&+&\sum_{(vc)\neq\ell }\frac{1}{(\omega_{cv}^{S})^{3}}\Big(\frac{\mbox{Im}\lbrack
v_{\ell c}^{\Sigma ,a}\{v_{cv}^{\Sigma ,b}v_{v\ell }^{\Sigma ,c}\}]}{\omega
_{c\ell }^{S}-2\omega_{cv}^{S}}-\frac{\mbox{Im}\lbrack v_{v\ell }^{\Sigma
,a}\{v_{\ell c}^{\Sigma ,b}v_{cv}^{\Sigma ,c}\}]}{\omega_{\ell v}^{S}-2\omega
_{cv}^{S}}\Big)\delta(\omega_{cv}^{S}-\omega)  \notag \\
&-&\sum_{vc}\frac{1}{(\omega_{cv}^{S})^{3}}\Big(4\mbox{Re}\lbrack \
v_{vc}^{\Sigma ,a}\{\mathcal{F}_{cv}^{bc}\}]\delta(\omega_{cv}^{S}-2\omega)+\mbox{Re}
\lbrack \{\mathcal{F}_{vc}^{ab}v_{cv}^{\Sigma ,c}\}]\delta(\omega_{cv}^{S}-\omega)
\Big)\Big],
\end{eqnarray}
as the imaginary part of the nonlinear SHG susceptibility for the scissored
Hamiltonian within the velocity gauge formalism, where we have used the
subscript $\mathrm{v}$ to denote it. 
The $\mathrm{Re}[\chi^{abc}_{\mathrm{v}}(-2\omega;\omega,\omega)]$ is
obtained through the Kramers-Kroning transformation.
Taking $\Delta =0$ we get 
\begin{eqnarray}
\mbox{Im}\lbrack \chi _{\mathrm{v},\Delta=0}^{abc}(-2\omega;\omega,\omega)] &=&\frac{\pi
|e|^{3}}{2\hbar ^{2}}\int \frac{d^{3}k}{8\pi ^{3}}\Big[\sum_{vc}\frac{16}{(
\omega_{cv})^{3}}\Big(\sum_{c^{\prime }}\frac{\mbox{Im}\lbrack
v_{vc}^{a}\{v_{cc^{\prime }}^{b}v_{c^{\prime }v}^{c}\}]}{\omega_{cv}-2\omega
_{c^{\prime }v}}  \notag  \label{imchid0} \\
&-&\sum_{v^{\prime }}\frac{\mbox{Im}\lbrack v_{vc}^{a}\{v_{cv^{\prime
}}^{b}v_{v^{\prime }v}^{c}\}]}{\omega_{cv}-2\omega_{cv^{\prime }}}\Big)\delta(\omega
_{cv}-2\omega)  \notag \\
&+&\sum_{(vc)\neq\ell }\frac{1}{(\omega_{cv})^{3}}\Big(\frac{\mbox{Im}\lbrack v_{\ell
c}^{a}\{v_{cv}^{b}v_{v\ell }^{c}\}]}{\omega_{c\ell }-2\omega_{cv}}-\frac{\mbox{Im}
\lbrack v_{v\ell }^{a}\{v_{\ell c}^{b}v_{cv}^{c}\}]}{\omega_{\ell v}-2\omega_{cv}}
\Big)\delta(\omega_{cv}-\omega)\Big],
\end{eqnarray}
since $\mathcal{F}_{nm}^{ab}|_{\Delta=0}=0$ (see the appendix). This
equation is identical to one obtained earlier for the unscissored
Hamiltonian.\cite{ghahramaniPRB91} However,
as far as we know the expression for 
$\mbox{Im}[\chi ^{abc}(-2\omega;\omega,\omega)]$ 
given in Eq.~\eqref{imchicf} is new, and the last two terms
proportional to $\Delta$ through $\mathcal{F}_{nm}^{ab}$ have been neglected in
the literature until now. As we show below (Sec. \ref{results}), these terms
are crucial for the gauge invariance of the second order response within the
scissored Hamiltonian.

Indeed, in the past the scissors implementation within the velocity gauge
has been performed by taking Eq.~\eqref{imchid0} and simply replacing $\omega
_{mn}$ by $\omega_{mn}^{S}$ and $\mathbf{v}_{mn}$ with $\mathbf{v}_{mn}^{\Sigma}$, as the
usual scissoring of the linear response
would (wrongly) suggest. This strategy leads to 
\begin{eqnarray}
\mbox{Im}\lbrack \chi _{\mathrm{v,wrong}}^{abc}(-2\omega;\omega,\omega)] &=&\frac{\pi
|e|^{3}}{2\hbar ^{2}}\int \frac{d^{3}k}{8\pi ^{3}}\Big[\sum_{vc}\frac{16}{(
\omega_{cv}^{S})^{3}}\Big(\sum_{c^{\prime }}\frac{\mbox{Im}\lbrack
v_{vc}^{\Sigma ,a}\{v_{cc^{\prime }}^{\Sigma ,b}v_{c^{\prime }v}^{\Sigma
,c}\}]}{\omega_{cv}^{S}-2\omega_{c^{\prime }v}^{S}}  \notag  \label{wrong} \\
&-&\sum_{v^{\prime }}\frac{\mbox{Im}\lbrack v_{vc}^{\Sigma
,a}\{v_{cv^{\prime }}^{\Sigma ,b}v_{v^{\prime }v}^{\Sigma ,c}\}]}{\omega
_{cv}^{S}-2\omega_{cv^{\prime }}^{S}}\Big)\delta(\omega_{cv}^{S}-2\omega)  \notag \\
&+&\sum_{(vc)\neq\ell }\frac{1}{(\omega_{cv}^{S})^{3}}\Big(\frac{\mbox{Im}\lbrack
v_{\ell c}^{\Sigma ,a}\{v_{cv}^{\Sigma ,b}v_{v\ell }^{\Sigma ,c}\}]}{\omega
_{c\ell }^{S}-2\omega_{cv}^{S}}-\frac{\mbox{Im}\lbrack v_{v\ell }^{\Sigma
,a}\{v_{\ell c}^{\Sigma ,b}v_{cv}^{\Sigma ,c}\}]}{\omega_{\ell v}^{S}-2\omega
_{cv}^{S}}\Big)\delta(\omega_{cv}^{S}-\omega)\Big],
\end{eqnarray}
a wrong result, since we are missing the important contribution from 
$\mathcal{F}_{mn}^{ab}$ given in Eq.~\eqref{imchicf}. It is obvious that the
coincidence that takes place in the linear response does not arise here at
all, since if we substitute $\mathbf{v}_{nm}^{\Sigma}=(\omega_{nm}^{S}/\omega_{nm})\mathbf{v}_{nm}
$ in Eq.~\eqref{wrong} we do not get the last two terms on the right hand
side of Eq.~\eqref{imchicf}!


\subsection{Length gauge Formalism}

Within this gauge, the interaction Hamiltonian is given by 
\begin{equation}
H_{I}(t)=-e\mathbf{r}\cdot \mathbf{E}(t).  \label{rde}
\end{equation}
As discussed in Nastos et al.\cite{nastosPRB05}, the
length-gauge formalism for the scissored Hamiltonian can be easily worked
out by simply using the unscissored Hamiltonian for the unperturbed system
with $-e\mathbf{r}\cdot \mathbf{E}(t)$ as the interaction, and then at the end of the
calculation only replacing $\omega_{nm}$ by $\omega_{nm}^{S}$ to obtain the
scissored results for any susceptibility expression, whether linear or
nonlinear. Indeed, $\mathbf{r}_{nm}$ and $\mathbf{r}_{nm;\mathbf{k}}$, as stated before, are
calculated within the unscissored (LDA) Hamiltonian. We use 
$
H(t)=H_{0}-e\mathbf{r}\cdot \mathbf{E}(t)  \label{hamlong}
$
as the time dependent Hamiltonian, that from Eq.~\eqref{mv} gives
$\dot{\mathbf{r}}=\mathbf{v}=\mathbf{p}/m_e$.

Taking the matrix elements of Eq.~\eqref{rhop} but now with the $H_{I}(t)$ of
Eq.~\eqref{rde}, we obtain 
$(\tilde{\rho}_{L}^{(1)}(t))_{nm}=B_{nm}^{b}E^{b}(\omega)e^{i(\omega_{nm}-\tilde\omega)t}$,
with 
\begin{equation*}
B_{nm}^{b}=\frac{e}{\hbar }\frac{f_{mn}r_{nm}^{b}}{\omega_{nm}-\tilde\omega},
\label{rho11}
\end{equation*}
and 
\begin{eqnarray*}
(\tilde{\rho}_{L}^{(2)}(t))_{nm} &=&\frac{e}{i\hbar }\frac{1}{\omega_{nm}-2\tilde\omega}\bigg[
i\sum_{\ell }\Big(r_{n\ell }^{b}B_{\ell m}^{c}-B_{n\ell }^{c}r_{\ell m}^{b}
\Big)  \notag 
-(B_{nm}^{c})_{;k^{b}}\bigg]E^{b}(\omega)E^{c}(\omega)e^{i(\omega_{nm}-2\tilde\omega)t}
.
\end{eqnarray*}
We have used the fact that for a cold semiconductor $\partial
f_{n}/\partial \mathbf{k}=0$ and thus the intraband contribution to the linear
term vanishes identically. 
From Eq.~\eqref{pe} we can obtain\cite{aversaPRB95} 
\begin{eqnarray*}\label{chie}
\chi _{L,e}^{abc}(-2\omega;\omega,\omega)
&=&
\frac{e^{3}}{\hbar^{2}}\int \frac{d^{3}k}{8\pi ^{3}}
\Big[
\sum_{\ell(n\neq m)}
\frac{2f_{nm}}{\omega_{mn}-2\tilde\omega}
+
\sum_{m(\ell\neq n)}
\frac{f_{\ell n}}{\omega_{\ell n}-\tilde\omega}
\nonumber\\
&+&
\sum_{n(\ell\neq m)}
\frac{f_{m\ell }}{\omega_{m\ell }-\tilde\omega}
\Big]
\frac{r_{nm}^{a}\{r_{m\ell }^{b}r_{\ell n}^{c}\}}{\omega_{\ell n}-\omega_{m\ell }}
,
\end{eqnarray*}
as the contribution from interband processes only, and 
\begin{eqnarray*}
\chi _{L,i}^{abc}(-2\omega;\omega,\omega) &=&\frac{ie^{3}}{\hbar ^{2}}\int \frac{
d^{3}k}{8\pi ^{3}}\sum_{m\neq n}f_{nm}\Big[\frac{2r_{nm}^{a}\{r_{mn;k^{c}}^{b}\}}{
\omega_{mn}(\omega_{mn}-2\tilde\omega)}+\frac{\{r_{nm;k^{c}}^{a}r_{mn}^{b}\}}{\omega_{mn}(\omega
_{mn}-\tilde\omega)}  \notag  \label{chii} \\
&+&\frac{1}{\omega_{mn}^{2}}\Big(\frac{1}{\omega_{mn}-\tilde\omega}-\frac{4}{\omega_{mn}-2\tilde\omega
}\Big)r_{nm}^{a}\{r_{mn}^{b}{\cal V}_{mn}^{c}\}-\frac{
\{r_{nm;k^{a}}^{b}r_{mn}^{c}\}}{2\omega_{mn}(\omega_{mn}-\tilde\omega)}\Big],
\end{eqnarray*}
as the contribution from intraband processes only, where $r_{nm;k^{a}}^{b}$
is the generalized derivative of $\mathbf{r}$, 
and is explicitly given by,\cite{aversaPRB95} 
\begin{equation}\label{rgen}
r_{nm;k^{a}}^{b}=\frac{r_{nm}^{a}{\cal V}_{mn}^{b}+r_{nm}^{b}{\cal V}_{mn}^{a}}{\omega
_{nm}}+\frac{i}{\omega_{nm}}\sum_{\ell }\big(\omega_{\ell m}r_{n\ell }^{a}r_{\ell
m}^{b}-\omega_{n\ell }r_{n\ell }^{b}r_{\ell m}^{a}\big)\quad (n\neq m),
\end{equation}
where ${\cal V}_{nm}^{a}=(p_{nn}^{a}-p_{mm}^{a})/m_e$ is the difference between the
electron velocity at bands $n$ and $m$, and the sum over $\ell $ is over all
the valence and conduction states. 

We notice that the above expressions for $\chi _{L,e,i}^{abc}(-2\omega;\omega,\omega)$ are
divergence free at $\omega=0$, that both satisfy the intrinsic permutation
symmetry $\chi _{L,e,i}^{abc}(-2\omega;\omega,\omega)=\chi _{L,e,i}^{acb}(-2\omega;\omega,\omega)$, and finally that the
full susceptibility 
$\chi _{L}^{abc}(-2\omega;\omega,\omega)=\chi _{L,e}^{abc}(-2\omega;\omega,\omega)+\chi _{L,i}^{abc}(-2\omega;\omega,\omega)$,
where the subscript $L$ denotes the length gauge. Again using time reversal
symmetry, we can take $\mathbf{r}_{mn}(\mathbf{k})=\mathbf{r}_{nm}(-\mathbf{k})$ and $\mathbf{r}_{mn;\mathbf{k}}(
\mathbf{k})=-\mathbf{r}_{nm;\mathbf{k}}(-\mathbf{k})$, along with the hermiticity condition $\mathbf{r}%
_{mn}=\mathbf{r}_{nm}^{\ast }$, which implies that $\mathbf{r}_{mn;\mathbf{k}}=\mathbf{r}_{nm;\mathbf{k}%
}^{\ast }$, and arrive at the following results for the imaginary parts of $
\chi _{i,e}^{abc}$, 
\begin{eqnarray}
\mbox{Im}\lbrack \chi _{L,e}^{abc}] &=&\frac{\pi |e|^{3}}{\hbar ^{2}}\int 
\frac{d^{3}k}{8\pi ^{3}}\sum_{(vc)\neq \ell }\Big[\frac{2\mbox{Re}\lbrack
r_{vc}^{a}\{r_{c\ell }^{b}r_{\ell v}^{c}\}]}{\omega_{c\ell }^{S}-\omega_{\ell
v}^{S}}\delta(\omega_{cv}^{S}-2\omega)  \notag  \label{cual} \\
&+&\Big(\frac{\mbox{Re}\lbrack r_{v\ell }^{a}\{r_{\ell c}^{b}r_{cv}^{c}\}]}{
\omega_{cv}^{S}-\omega_{\ell c}^{S}}+\frac{\mbox{Re}\lbrack r_{\ell
c}^{a}\{r_{cv}^{b}r_{v\ell }^{c}\}]}{\omega_{v\ell }^{S}-\omega_{cv}^{S}}\big)\delta(
\omega_{cv}^{S}-\omega)\Big]
\end{eqnarray}
and 
\begin{eqnarray}
\mbox{Im}\lbrack \chi _{L,i}^{abc}] &=&\frac{\pi |e|^{3}}{\hbar ^{2}}\int 
\frac{d^{3}k}{8\pi ^{3}}\sum_{vc}\Big[\Big(\frac{2\mbox{Im}\lbrack
r_{vc}^{a}\{r_{cv;k^{c}}^{b}\}]}{\omega_{cv}^{S}}-\frac{4\mbox{Im}\lbrack
r_{vc}^{a}\{r_{cv}^{b}{\cal V}_{cv}^{c}\}]}{(\omega^{S})_{cv}^{2}}\Big)\delta(\omega
_{cv}^{S}-2\omega)  \notag  \label{imchii2} \\
&+&\Big(\frac{\mbox{Im}\lbrack \{r_{vc;k^{c}}^{a}r_{cv}^{b}\}]}{\omega_{cv}^{S}}
+\frac{\mbox{Im}\lbrack r_{vc}^{a}\{r_{cv}^{b}{\cal V}_{cv}^{c}\}]}{(\omega
^{S})_{cv}^{2}}-\frac{\mbox{Im}\lbrack \{r_{vc;k^{a}}^{b}r_{cv}^{c}\}]}{2\omega
_{cv}^{S}}\Big)\delta(\omega_{cv}^{S}-\omega)\Big],
\end{eqnarray}
where we have taken $\omega_{nm}\rightarrow \omega_{nm}^{S}$ so the above
expressions are valid for the scissors Hamiltonian, $H^{S}=H_{0}+S(\mathbf{r},\mathbf{p}
)-e\mathbf{r}\cdot \mathbf{E}$. Recall that both $\mathbf{r}_{nm}$ and $\mathbf{r}_{nm;\mathbf{k}}$ are
calculated with the unscissored (Kohn-Sham) Hamiltonian.\cite{nastosPRB05}
The last two equations give the nonlinear SHG susceptibility within the
length gauge for the scissored Hamiltonian. 
Comparing
Eq.~\eqref{cual} and Eq.~\eqref{imchii2} with the velocity gauge result of
Eq.~\eqref{imchicf} it is clear that, unlike for linear response, there is
no obvious analytical scheme to prove that both gauges give the same result.
In the following section we present numerical results to prove the expected
gauge invariance.

%%%%%%%%%%%%%%%%%%%%%%%%%%%%%%%%%%%%%%%%%%%%%%%%%%%%%%%%%%%%%%%%%%

\section{Results}

\label{results}

In this section we evaluate the velocity and length gauge expressions 
$\chi _{\mathrm{v},L}^{abc}(-2\omega;\omega,\omega)$ for GaAs.
In
order to calculate the energies, wave functions and matrix elements we
employ the ``augmented plane wave plus local orbital method'' using the WIEN2K
code. \cite{wien2k}
 This all-electron code uses the full local-crystal
potential, i.e. $V(\mathbf{r})$, just as required by our assumptions of 
Eq.~\eqref{h0}, and thus the commutator $[\mathbf{r},H]=i\hbar
\dot{\mathbf{r}}$
is correctly calculated for the local $V(\mathbf{r})$. We also
show results calculated through the use of pseudo-potentials with the ABINIT
plane-wave code.\cite{gonzeCMS02} The pseudo-potentials are non-local
functions expressible as $V^{\mathrm{NL}}(\mathbf{r},\mathbf{p})$,\cite{staracePRA71}
just as is the scissor Hamiltonian $S(\mathbf{r},\mathbf{p})$. To really complete
the calculation one would have to do the corresponding manipulations
including $V^{\mathrm{NL}}(\mathbf{r},\mathbf{p})$ in the Hamiltonian $H^{S}$ 
(Eq.~\eqref{hsa}), and new terms would arise in the linear and nonlinear
susceptibility expressions. For instance, the term 
$i\hbar \mathbf{v}^{\mathrm{NL}}=[\mathbf{r},V^{\mathrm{NL}}(\mathbf{r},\mathbf{p})]$ should be added to the
velocity operator $\mathbf{v}^{\Sigma}$ given by Eq.~\eqref{vs}. This is a research
project for the future. 
Here we use the
comparison of the all-electron and the pseudopotential calculation
to get a sense of the size of error involved by neglecting
the nonlocal contributions coming from
$V^{\mathrm{NL}}(\mathbf{r},\mathbf{p})$. 
The
linear response counterpart of these calculations are 
discussed in Pulci et al.\cite{pulciPRB98} and Mendoza et al.\cite{mendozaPRB06} 

\begin{table}[b]
\begin{center}
\begin{tabular}{|r|c|c|}
\hline
Parameter (GaAs) & all-electron & pseudopotential \\ \hline
Lattice parameter & 10.684$a_0$ & 10.684$a_0$ \\ 
$\mathbf{k}$-points  & 27720 & 27720 \\ 
Unscissored band gap & 0.277 eV & 0.469 eV \\ 
Scissors & 1.243 eV & 1.051 eV \\ 
Valence bands & 14 (includes semi-core) & 4 \\ 
Conduction bands & 7 & 7 \\ 
Exchange correlation energy & LDA & LDA \\ 
Energy convergence limit & 0.001 Ry & - \\ 
Cut-off energy & - & 20 Ha \\ 
$R_{\mathrm{MT}}K_{\mathrm{MAX}}$ & 7.0 & - \\ \hline
\end{tabular}
\end{center}
\caption{The most important parameters used in the all-electron and
pseudopotential schemes for GaAs. The empty entries are not relevant for the
corresponding code. The $\mathbf{k}$-points are for the irreducible part of
the first Brillouin zone, and $R_{\mathrm{MT}}K_{\mathrm{MAX}}$
is a product of the ``muffin-tin'' radius $R$ 
and the maximum value for the plane wave vectors $\mathbf{K}$.\cite{wien2k} 
}
\label{tabla}
\end{table}

Spin-orbit effects, local field effects, and the consequences of the
electron-hole attraction\cite{leitsmannPRB05} on the SHG process are
neglected. Although all these effects are important for the optical
response of a semiconductor, their calculation is still an open
question and a numerical challenge that ought to be pursued. However
this endeavour is
beyond the scope of this article.
The band gap of GaAs is taken to be its experimental value of
1.52 eV. We find converged spectra for all the quantities of interest in
this work, and the most important parameters are shown in the table 
~\ref{tabla}. All the spectra are calculated with an energy smearing of 0.15 eV.
The linear analytic tetrahedron method is used to evaluate the Brillouin
zone integrals for the imaginary part of the spectra, where special care was
taken to examine the double resonances.\cite{nastosPRB05} Double resonances
occur if for a given frequency $\omega$ there can be resonant transitions at
both frequencies $\omega$ and $2\omega$, that is, if there is a region in the
Brillouin zone such that $\omega_{cv}(\mathbf{k})=2\omega_{c^{\prime }v}(\mathbf{k})=2\omega$.
For these $\mathbf{k}$ points the perturbation theory used to calculate the
spectrum breaks down, since there is real population excited, which in a
correct calculation must be taken into account. These points introduce sharp
spikes in the spectrum that can in principle affect the low-frequency
results, since the response at frequencies below the band gap is computed
from the Kramers-Kronig relation. However, in agreement with 
Nastos et al.,\cite{nastosPRB05} we find here that the double
resonances affect the low-frequency results by less than 2\%.

\begin{figure}[t]
\includegraphics[scale=.7]{fig1}
\caption{(a) Im[$\protect\chi
^{xyz}_{\mathrm{v},L}(-2\omega;\omega,\omega)$] for the length and the velocity gauge schemes, using
the all-electron approach and for zero scissors correction,
$\Delta=0$. 
(b)
$\mbox{Im}\lbrack
\protect\chi^{xyz}_{L}(-2\omega;\omega,\omega)-\protect\chi^{xyz}_{\mathrm{v}}(-2\omega;\omega,\omega)]$ 
where very tiny
differences between the two schemes are seen.
}
\label{fimd0}
\end{figure}

In Fig.~\ref{fimd0} we show the imaginary part of $\chi _{\mathrm{v}
,L}^{xyz}(-2\omega;\omega,\omega) $ with no scissors correction ($\Delta=0$), calculated with the
all-electron scheme.\cite{nota1} As expected, $\mbox{Im}\lbrack \chi _{
\mathrm{v},L}^{xyz}](-2\omega;\omega,\omega)$ is zero below the gap, and above it we see a series of
positive and negative peaks that can be related to electronic transitions.
What is more relevant for this article is that in the top panel we have
plotted both $\chi _{\mathrm{v}}^{xyz}(-2\omega;\omega,\omega)$ and $\chi _{L}^{xyz}(-2\omega;\omega,\omega)$; they seem
identical, as they must be, since gauge invariance must be fulfilled. In the
bottom panel of Fig.~\ref{fimd0}, we show $\mbox{Im}\lbrack \chi
_{L}^{xyz}(-2\omega;\omega,\omega)-\chi _{\mathrm{v}}^{xyz}(-2\omega;\omega,\omega)]$,
which confirms that the results for the unscissored 
$\mathrm{Im}[\chi _{\mathrm{v},L}^{abc}(-2\omega;\omega,\omega)]$ 
agree to within numerical accuracy (about 1 part in approximately
$10^5$),
as would be expected from gauge invariance.

\begin{figure}[t]
\includegraphics[scale=.7]{fig2}
\caption{(color on line) 
(a) Im[$\protect\chi^{xyz}_{\mathrm{v}}(-2\omega;\omega,\omega)$] and Im[$\protect\chi
^{xyz}_{\mathrm{v,wrong}}(-2\omega;\omega,\omega)$] for the velocity
gauge.
(b)
Im[$\protect\chi^{xyz}_L (-2\omega;\omega,\omega)$] for the
longitudinal gauge.
(c)
Im[$\protect\chi
^{xyz}_{L}(-2\omega;\omega,\omega)-\protect\chi^{xyz}_{\mathrm{v}}(-2\omega;\omega,\omega)$] where very tiny
differences are seen. The spectra is evaluated within the all-electron
approach with $\Delta=1.243$ eV.}
\label{fimd}
\end{figure}

\begin{figure}[t]
\includegraphics[scale=1]{fig3}
\caption{(color on line)
$|\protect\chi^{xyz}_{L}(-2\omega;\omega,\omega)|=|\protect\chi^{xyz}_{\mathrm{v}}(-2\omega;\omega,\omega)|
\equiv|\protect\chi^{xyz}(-2\omega;\omega,\omega)|$ with a scissors correction of $\Delta=1.243$ eV for the
all-electron calculation and $\Delta=1.051$ eV for the pseudopotential
calculation.}
\label{flvsv}
\end{figure}

In Fig.~\ref{fimd} we show the imaginary part of $\chi _{\mathrm{v},L}^{xyz}(-2\omega;\omega,\omega)$
with a scissors shift of $\Delta=1.243$ eV, calculated with the all-electron
scheme. In the top panel we compare the velocity-gauge calculation
(i.e. $\mathrm{Im}[\chi _{\mathrm{v}}^{xyz}(-2\omega;\omega,\omega)]$ of
Eq.~\eqref{imchicf})
 with
a calculation where we neglect the new contributions coming from the
scissors term, (i.e. $\mathrm{Im}[\chi
_{\mathrm{v,wrong}}^{xyz}(-2\omega;\omega,\omega)]$ of Eq.~\eqref{wrong});
 we see that the results disagree. In the middle
panel we show $\mbox{Im}\lbrack \chi _{L}^{xyz}(-2\omega;\omega,\omega)]$ and in the bottom panel we
show $\mbox{Im}\lbrack \chi _{L}^{xyz}(-2\omega;\omega,\omega)-\chi
_{\mathrm{v}}^{xyz}(-2\omega;\omega,\omega)]$, 
where it is clear that, as in the 
unscissored case, gauge invariance with the scissored Hamiltonian is confirmed 
within numerical accuracy.
We stress that this
fulfillment of gauge invariance is due to the new terms of 
Eq.~\eqref{imchicf} proportional to $\mathcal{F}_{mn}^{ab}$ which in turn
depends on the commutator $[\mathbf{r},\mathbf{v}^{S}]$ with $\mathbf{v}^{S}=-(i/\hbar )[\mathbf{r}%
,S(\mathbf{r},\mathbf{p})]$. Thus, neglecting the effect of the scissors operator 
$S(\mathbf{r},\mathbf{p})$ in the usual perturbation procedure would lead, in general, to
the wrong result for nonlinear susceptibility tensors within the velocity
gauge approach.

As explained above, we have also used a pseudo-potential method to calculate
the SHG susceptibility tensor. In this way, we can estimate the error that
one makes when calculating the matrix elements of the electron's momentum
operator through the use of pseudopotentials, the error arising from the
non-local part of the pseudopotential in the 
commutators.\cite{pulciPRB98,mendozaPRB06} 
In Fig.~\ref{flvsv} we show the absolute value of $
|\chi _{L}^{xyz}(-2\omega;\omega,\omega)|=|\chi
_{\mathrm{v}}^{xyz}(-2\omega;\omega,\omega)|\equiv |\chi
^{xyz}(-2\omega;\omega,\omega)|$
 with the
scissors correction, where $\Delta=1.051$ eV for the pseudopotential code and $
\Delta=1.243$ eV for the all-electron code. We notice that there is a
difference between the results of the value of the static limit of
$|\chi^{xyz}(-2\omega;\omega,\omega)|$
 by approximately 36.8 pm/V; we obtain a static value of $|\chi^{xyz}(0;0,0)|=135.6$ pm/V for the pseudo-potential calculation, and $|\chi^{xyz}(0;0,0)|=172.4$ pm/V, for the all-electron calculation. These quantities
are close to the theoretical values of other studies,\cite{nastosPRB05} and
to the most recent experimental value close to the static limit of 172
pm/V at 0.118 eV.\cite{eyresAPL01} We see that the corrections due to the
non-local nature of the pseudopotentials affect not only the strength of the
spectrum but also its line shape, as some resonances are energy shifted from
one calculation to the other. The overall intensity correction is smaller
than $\sim 25\%$, and we may conclude that the pseudopotential
calculation does a reasonable job for the nonlinear response. Indeed, this
seems to be the case for the linear optical response 
as well.\cite{mendozaPRB06}

\begin{figure}[t]
\includegraphics[scale=1]{fig4}
\caption{(color on line)
$|\chi^{xyz}_{L}|=|\chi^{xyz}_{v}|=\chi^{xyz}$ with with a scissors correction of
$\Delta=1.243$ eV
for the
all-electron calculation,
along with the
experimental results of
Ref.~\onlinecite{bergfeldPRL03}.
The top energy scale is the original energy $E_{\mathrm{orig}}$
of the all-electron calculation.
The bottom scale is the scaled energy, $2E_{\mathrm{mod}}$,\cite{nota3} for the theoretical
spectra and the two-photon energy of the experimental results.
}
\label{exp}
\end{figure}

Although the main objective of this article is to show how the
non-local scissors correction must be included in the lineal
and non-lineal optical response, and how including it fulfills
gauge invariance, as shown in Fig.~\ref{fimd}, we present for reference
the comparison of the theoretical results with the experimentally
results.   
In Fig.~\ref{exp} we  show the experimental spectrum measured by
Bergfeld and Daum,\cite{bergfeldPRL03} where in order to have a better
comparison of theory and experiment, the energy scale
of the theoretical results has been linearly rescaled as proposed 
by them.\cite{nota3}
Our results for the all-electron calculation
show  good agreement with the experimental values up to 4.3 eV.
Above 4.3 eV
the theoretical signal disagrees although it shows a similar line
shape that is blue-shifted in energy with respect to the experimental signal.
We have checked that the results obtained in 
Ref.~\onlinecite{rashkeevPRB98,leitsmannPRB05,nastosPRB05,adolphPRB98,hughesPRB96}
qualitatively show a similar comparison with the experiment.

%The quantitative differences may lay on the different
%parameters used in each calculation, like those of Table \ref{tabla}.



\section{Conclusions}

\label{conclusions}

We have presented a comparison for the calculation of the second harmonic
susceptibility tensor using two well-known approaches, often colloquially
referred to as using the ``velocity gauge'' and the ``length gauge''. We have
done this for two Hamiltonians, the usual LDA Hamiltonian and the scissored
Hamiltonian, where a rigid energy shift in the conduction bands is
introduced so the experimental (or GW) energy gap is obtained. We derived a
new expression for the velocity gauge susceptibility 
$\chi _{\mathrm{v}}^{abc}(-2\omega;\omega,\omega)$, where correction terms related to the non-local
nature of the scissors operator were obtained. These terms, not considered
before in the literature, are crucial in order to obtain gauge invariance
for a calculation made with the scissored Hamiltonian. For the unscissored
Hamiltonian, gauge invariance is obtained with the usual 
$\chi^{abc}(-2\omega;\omega,\omega)$ expression for the velocity and length gauge.

We have presented our numerical results for GaAs using a DFT-LDA {\it ab initio}
calculation, with the augmented plane wave plus local orbital all-electron
method as given by WIEN2K,\cite{wien2k} and a plane-wave pseudo-potential
scheme given by the ABINIT code.\cite{gonzeCMS02} 
Besides providing a numerical demonstration of gauge invariance for the 
unscissored and the scissored Hamiltonian calculations, this indicates the kind 
and size of error
that the neglect of the non-local nature of the pseudo-potentials can
be expected to produce in the calculation of 
$\chi_{\mathrm{v},L}^{abc}(-2\omega;\omega,\omega)$;
 it affects  not only the strength of the spectrum, but also
its line shape. Our results compare qualitatively well with the previous
work of other authors, and in particular with the experimental results.
However, the details of each approach show that the calculation of the
nonlinear response in a nontrivial matter, and better calculations of 
$\chi^{abc}(-2\omega;\omega,\omega)$ using more sophisticated means are still to be
sought. 

\section{Acknowledgments}

J.L.C. and M.E. acknowledges partial support through CONACyT. B.S.M. acknowledges
support by CONACYT 48915-F. F.N. acknowledges support by the Ontario
Graduate Scholarship program. J.E.S. acknowledges the National Science and
Engineering Research Council of Canada and Photonics Research Ontario.

\appendix

\section{}

In this Appendix we derive several results related to the scissors operator 
$S(\mathbf{r},\mathbf{p})$ of Eq.~\eqref{hats}. First we sketch some well known results,
for which we follow Aversa and Sipe,\cite{aversaPRB95}
 and 
Blount.\cite{blountSSP62} 
We write the position
operator of the electron, $\mathbf{r}$, as the sum of its \textit{interband} part 
$\mathbf{r}_{e}$ and \textit{intraband} part $\mathbf{r}_{i}$, $\mathbf{r}=\mathbf{r}_{e}+\mathbf{r}_{i}$.
The matrix elements of $\mathbf{r}_{e}$ are simply given by\cite{aversaPRB95} 
\begin{equation}
\langle n\mathbf{k}|\mathbf{r}_{e}|m\mathbf{k}'\rangle=\delta(\mathbf{k}-\mathbf{k}^{\prime })(\mathbf{r}
_{e})_{nm}\rightarrow \mathbf{r}_{nm}=\frac{\mathbf{p}_{nm}}{im\omega_{nm}}=\frac{\mathbf{v}_{nm}
}{i\omega_{nm}}\quad n\neq m,  \label{remn}
\end{equation}
where the canonical momentum matrix elements are calculated according to 
\begin{equation*}\label{pmn}
\langle n\mathbf{k}|\mathbf{p}|m\mathbf{k}'\rangle=\delta(\mathbf{k}-\mathbf{k}')\mathbf{p}_{nm}
=\delta(\mathbf{k}-\mathbf{k}')\int d^3r
\psi_{n\mathbf{k}}^*(\mathbf{r})(-i\hbar\boldsymbol{\nabla}) \psi_{m\mathbf{k}}(\mathbf{r})
,
\end{equation*}
and $\mathbf{v}_{nm}=\mathbf{p}_{nm}/m_e$. Instead of needing the matrix elements of $\mathbf{r}%
_{i}$ one actually uses its following property,\cite{aversaPRB95} 
\begin{equation}
\langle n\mathbf{k}|[\mathbf{r}_i,{\cal O}]|m\mathbf{k}'\rangle=i\delta(\mathbf{k}-\mathbf{k}^{\prime })(\mathcal{O}_{nm})_{;\mathbf{k}},  \label{re}
\end{equation}
where $\mathcal{O}$ is an operator and $(\mathcal{O}_{nm})_{;\mathbf{k}}$ is the
generalized derivative of its matrix elements, i.e. Eq.~\eqref{rgen}
 for $r_{nm;k^{a}}^{b}$. As discussed by Nastos et al.\cite{nastosPRB05},
 both $\mathbf{r}_{nm}$ (Eq.~\eqref{remn}), and its
generalized derivative $\mathbf{r}_{nm;\mathbf{k}}$ (Eq.~\eqref{rgen}), are evaluated
using the unscissored energies.

Now we establish Eq.~\eqref{vsn}. We take matrix elements of Eq.~\eqref{vnl}
and use Eq.~\eqref{hats} to write 
\begin{eqnarray}
\mathbf{v}_{nm}^{S} &=&-\frac{i}{\hbar }\langle n\mathbf{k}|(\mathbf{r} S(\mathbf{r},\mathbf{p})-S(\mathbf{r},\mathbf{p})
\mathbf{r})|m\mathbf{k}\rangle  \notag  \label{vnla} \\
&=&-i\Delta\Big((1-f_{m})-(1-f_{n})\Big)\langle n\mathbf{k}|\mathbf{r}|m\mathbf{k}\rangle  \notag \\
&=&i\Delta f_{mn}\mathbf{r}_{nm}  \notag \\
&=&\frac{\Delta f_{mn}}{m\omega_{nm}}\mathbf{p}_{nm},
\end{eqnarray}
where we used Eq.~\eqref{remn} since the factor, $f_{mn}$, yields $%
n\neq m$. Then the matrix elements of Eq.~\eqref{vs} reduce to 
\begin{eqnarray}
\mathbf{v}_{nm}^{\Sigma} &=&\big(1+\frac{\Delta f_{mn}}{\omega_{nm}}\big)\langle n\mathbf{k}|\frac{%
\mathbf{p}}{m}|m\mathbf{k}\rangle  \notag  \label{vsa} \\
&=&\big(\frac{\omega_{nm}+\Delta f_{mn}}{\omega_{nm}}\big)\mathbf{v}_{nm}=\big(\frac{\omega%
_{n}^{S}-\omega_{m}^{S}}{\omega_{nm}}\big)\mathbf{v}_{nm}  \notag \\
&=&\frac{\omega_{nm}^{S}}{\omega_{nm}}\mathbf{v}_{nm}\quad(n\neq m),
\end{eqnarray}
where we used Eq.~\eqref{wese}; thus Eq.~\eqref{vsa} is Eq.~\eqref{vsn}.

In order to prove Eq.~\eqref{calfnn}, we start with the matrix elements of
Eq.~\eqref{calefe}, which we write as 
\begin{equation*}
\mathcal{F}_{nm}^{ab}=\langle n\mathbf{k}|\big(\lbrack 
r_{i}^{a},v^{S,b}]+[r_{e}^{a},v^{S,b}]\big)|m\mathbf{k}\rangle.  
\end{equation*} 
The interband part is 
\begin{eqnarray}
\langle n\mathbf{k}| [r_e^a,v^{S,b}]|m\mathbf{k}\rangle &\equiv &\mathcal{F}
_{e,nm}^{ab}=\sum_{\ell }\Big(r_{e,n\ell }^{a}v_{\ell m}^{S,b}-v_{n\ell
}^{S,b}r_{e,\ell m}^{a}\Big)  \notag  \label{calefemna} \\
&=&
i\Delta\sum_{\ell\neq(mn)}\Big(f_{m\ell }r_{n\ell }^{a}r_{\ell
m}^{b}-f_{\ell n}r_{n\ell }^{b}r_{\ell m}^{a}\Big),
\end{eqnarray}
where we used Eq.~\eqref{remn} and Eq.~\eqref{vnla}. For the intraband
part we use the result of
Eq.~\eqref{re} and Eq.~\eqref{vnla}  to simply write 
\begin{equation}
\langle n\mathbf{k}| [r_i^a,v^{S,b}]|m\mathbf{k}\rangle\equiv \mathcal{F}
_{i,nm}^{ab}=iv_{nm;k^{a}}^{S,b}=\Delta f_{nm}r_{nm;k^{a}}^{b}.  \label{vnlgena}
\end{equation}
From Eq.~\eqref{calefemna} and Eq.~\eqref{vnlgena} we find 
\begin{equation}
\mathcal{F}_{nm}^{ab}=
i\Delta\sum_{\ell\neq(mn)}\Big(f_{m\ell }r_{n\ell }^{a}r_{\ell
m}^{b}-f_{\ell n}r_{n\ell }^{b}r_{\ell m}^{a}\Big)
+
\Delta f_{nm}r_{nm;k^{a}}^{b}.  \label{calfnmab}
\end{equation}
We see that for $n=m$ the intraband contribution $\mathcal{F}%
_{i,nn}^{ab}=0$, whereas the interband part reduces to 
\begin{equation}
\mathcal{F}_{nn}^{ab}=\mathcal{F}_{e,nn}^{ab}=i\Delta\sum_{m \neq n}f_{nm
}(r_{nm }^{a}r_{m n}^{b}+r_{nm }^{b}r_{m n}^{a}),
\label{calfnnape}
\end{equation}
giving Eq.~\eqref{calfnn}.

Now we take matrix elements of $[r^{a},\mathcal{F}^{bc}]$, separating $%
r^{a}=r_{i}^{a}+r_{e}^{a}$. Then the interband part gives 
\begin{equation}\label{inter}
\lbrack r_{e}^{a},\mathcal{F}^{bc}]_{nn}=\sum_{m\neq n}\big(r_{nm}^{a}
\mathcal{F}_{mn}^{bc}-\mathcal{F}_{nm}^{bc}r_{mn}^{a}\big),
\end{equation}
while the intraband part gives 
\begin{equation}\label{intra}
\lbrack r_{i}^{a},\mathcal{F}^{bc}]_{nn}=i\mathcal{F}_{nn;a}^{bc}=i\frac{
\partial }{\partial k^{a}}\mathcal{F}_{nn}^{bc}=i\frac{\partial }{\partial
k^{a}}\mathcal{F}_{e,nn}^{bc},
\end{equation}
where we used Eq.~\eqref{calfnnape}. Then Eq.~\eqref{inter} and 
Eq.~\eqref{intra} are used to obtain Eq.~\eqref{tr33bb}. 

%%%%%
We derive Eq.~\eqref{lastb}.
Using
Eq.~\eqref{vsa},
$\omega_{mn}^{S}=\omega_{mn}-f_{mn}\Delta$, Eq.~\eqref{remn} and
Eq.~\eqref{calfnn},
 Eq.~\eqref{trd2} reduces to 
\begin{eqnarray}
\zeta ^{ab} &=&
\frac{e^{2}}{\hbar }\int \frac{d^{3}k}{8\pi ^{3}}
\sum_{m\ne n}f_n\omega_{mn}^S\frac{v_{nm}^av_{mn}^b
+v_{mn}^av_{nm}^b}{\omega_{mn}^2}-\frac{e^{2}n}{m}\delta_{ab}
\nonumber\\
&=&
\frac{e^{2}}{\hbar }\int \frac{d^{3}k}{8\pi ^{3}}
\sum_{m\ne n}f_n\omega_{mn}\frac{v_{nm}^av_{mn}^b
+v_{mn}^av_{nm}^b}{\omega_{mn}^2}
-\frac{e^{2}n}{m}\delta_{ab}
\nonumber\\
&-&
\frac{e^{2}\Delta}{\hbar }\int \frac{d^{3}k}{8\pi ^{3}}
\sum_{m\ne n}f_nf_{mn}\frac{v_{nm}^av_{mn}^b
+v_{mn}^av_{nm}^b}{\omega_{mn}^2}
\nonumber \\
&=&
-e^{2}\int \frac{d^{3}k}{8\pi ^{3}}
\sum_{n} 
f_n\Big(
\frac{\delta_{ab}}{m}
-
\sum_{m\ne n}
\frac{v_{nm}^av_{mn}^b
+v_{mn}^av_{nm}^b}{\hbar\omega_{mn}}
\Big)
\nonumber\\
&+&
\frac{e^{2}\Delta}{\hbar }\int \frac{d^{3}k}{8\pi ^{3}}
\sum_nf_n\sum_{m\ne n}f_{nm}\Big(r_{nm}^ar_{mn}^b
+r_{mn}^ar_{nm}^b\Big)
\nonumber\\
&=&-e^{2}\int \frac{d^{3}k}{8\pi ^{3}}\sum_{n}f_{n}\left[ \frac{1}{
m_{n}^{\ast}}\right]^{ab}
-
\frac{ie^{2}}{\hbar }\int \frac{d^{3}k}{8\pi
^{3}}\sum_{n}f_n{\cal F}^{ab}_{nn}
,
\label{trdn}
\end{eqnarray}
where 
\begin{equation}\label{meff}
\left[ \frac{1}{m_{n}^{\ast }}\right]
^{ab}=\frac{\delta_{ab}}{m}-\sum_{m\ne n}
\frac{v_{nm}^{a}v_{mn}^{b}+v_{mn}^{a}v_{nm}^{b}}{\hbar \omega_{mn}}
,
\end{equation}
is the effective mass tensor. Identifying the second term on the right hand
side of Eq.~\eqref{trdn} as $-\eta ^{ab}$ (see Eq.~\eqref{eta}), 
leads to 
\begin{equation*}
\zeta ^{ab}+\eta ^{ab}=-e^{2}\int \frac{d^{3}k}{8\pi ^{3}}\sum_{n}f_{n}\left[
\frac{1}{m_{n}^{\ast }}\right] ^{ab}.  \label{etam}
\end{equation*}
Using above in Eq.~\eqref{chi2} gives Eq.~\eqref{lastb}.

\bibliography{biblio}
\end{document}
